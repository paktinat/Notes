%From figure ~\ref{fig:nbjetPresel} one can clearly see how 
This section describes the samples used to control the $\tt$ background.

The $\tt$ process is in general dominant in the 3-jets samples with 1 or more b-tags, and is also the main background to the $t$-channel in the 2-jets 1-tag category. Two meaningful $\ttbar$-enriched control samples are therefore the 3-jets, 1-tag and 3-jets, 2-tags samples.

%This sample is also enriched in W+heavy flavor jets.


%The highest TCHP tagged jet is used as candidate for the top reconstruction for both samples.
%Tables~\ref{tab:3J1T} and~\ref{tab:3J2T} show the yields for simulation and data in this control sample.
%One can notice a difference of O(10$\%$) in the 3-jets 1-tag category. This difference is partially cured taking into account the data 
%driven scale factors for the \wjets component of the background (see also appendix). 
%
%\input{yieldTable3J1T}
%\input{yieldTable3J2T}
%
%Figures ~\ref{fig:3JNTeta}, ~\ref{fig:3JNTtopMass} show the light jet $\eta$ and the \mt in those control samples for muons (a,c) and electrons (b,d).
%
%\begin{figure}[!h]
%\begin{center}
%\subfigure[]{
%\includegraphics[angle=0,width=0.48\textwidth]{figures/3J_1T_noSyst_etalj_MuStack.png}}
%	    \subfigure[]{
%\includegraphics[angle=0,width=0.48\textwidth]{figures/3J_1T_noSyst_etalj_EleStack.png}}\\
%\vskip -2ex
%\subfigure[]{
%\includegraphics[angle=0,width=0.48\textwidth]{figures/3J_2T_noSyst_etalj_MuStack.png}}
%	    \subfigure[]{
%\includegraphics[angle=0,width=0.48\textwidth]{figures/3J_2T_noSyst_etalj_EleStack.png}}\\
%\end{center}
%\caption{\label{fig:3JNTeta} distribution of the light jet $\eta$ in the $\ttbar$-enriched samples in the 3-jets and 1(a,b) or 2(c,d) tags, 
%muons (a,c) and electrons (b,d). }
%%(a,c,e runs $\ge$ 166404, \lumiMu \invpb ) and electrons (b,d,f 166404 $\le$  runs $<$ \maxrun, \lumiEleValidation \invpb ) }
%\end{figure}
%
%\begin{figure}[!h]
%\begin{center}
%\subfigure[]{
%\includegraphics[angle=0,width=0.48\textwidth]{figures/3J_1T_noSyst_topMass_MuStack.png}}
%	    \subfigure[]{
%\includegraphics[angle=0,width=0.48\textwidth]{figures/3J_1T_noSyst_topMass_EleStack.png}}\\
%\vskip -2ex
%\subfigure[]{
%\includegraphics[angle=0,width=0.48\textwidth]{figures/3J_2T_noSyst_topMass_MuStack.png}}
%	    \subfigure[]{
%\includegraphics[angle=0,width=0.48\textwidth]{figures/3J_2T_noSyst_topMass_EleStack.png}}\\
%\end{center}
%\caption{\label{fig:3JNTtopMass} distribution of the \mt in the $\ttbar$-enriched samples in the 3-jets and 1(a,b) or 2(c,d) tags, 
%muons (a,c) and electrons (b,d). }
%%(a,c,e runs $\ge$ 166404, \lumiMu \invpb ) and electrons (b,d,f 166404 $\le$  runs $<$ \maxrun, \lumiEleValidation \invpb ) }
%\end{figure}
% 
%\ttbar backgrounds are expected to be symmetric, and this can be checked in data on those control samples. Figure \ref{fig:3JNTcharge} shows the lepton charge normalized to the data yield obtained for the sake of displaying this feature.
%
%\begin{figure}[!h]
%\begin{center}
%\subfigure[]{
%\includegraphics[angle=0,width=0.48\textwidth]{figures/3J_1T_noSyst_DDNorm_charge_MuStack.png}}
%	    \subfigure[]{
%\includegraphics[angle=0,width=0.48\textwidth]{figures/3J_1T_noSyst_DDNorm_charge_EleStack.png}}\\
%\vskip -2ex
%\subfigure[]{
%\includegraphics[angle=0,width=0.48\textwidth]{figures/3J_2T_noSyst_DDNorm_charge_MuStack.png}}
%	    \subfigure[]{
%\includegraphics[angle=0,width=0.48\textwidth]{figures/3J_2T_noSyst_DDNorm_charge_EleStack.png}}\\
%\end{center}
%\caption{\label{fig:3JNTcharge} lepton charge in the 3-jets and 1(a,b) or 2(c,d) tags, 
%muons (a,c) and electrons (b,d). Simulation normalized to the data yield.}
%%(a,c,e runs $\ge$ 166404, \lumiMu \invpb ) and electrons (b,d,f 166404 $\le$  runs $<$ \maxrun, \lumiEleValidation \invpb ) }
%\end{figure}
%
%Figg. ~\ref{fig:3JNTeta_pm} and  ~\ref{fig:3JNTtopMass_pm} show the behavior of the light jet $\eta$ and of the reconstructed top mass $\topMass$ for positive and negative charge leptons.
%
%\begin{figure}[!h]
%\begin{center}
%\subfigure[]{
%\includegraphics[angle=0,width=0.48\textwidth]{figures/3J_1T_noSyst_Plus_etalj_MuStack.png}}
%	    \subfigure[]{
%\includegraphics[angle=0,width=0.48\textwidth]{figures/3J_1T_noSyst_Plus_etalj_EleStack.png}}\\
%\subfigure[]{
%\includegraphics[angle=0,width=0.48\textwidth]{figures/3J_1T_noSyst_Minus_etalj_MuStack.png}}
%	    \subfigure[]{
%\includegraphics[angle=0,width=0.48\textwidth]{figures/3J_1T_noSyst_Minus_etalj_EleStack.png}}\\
%\vskip -2ex
%\subfigure[]{
%\includegraphics[angle=0,width=0.48\textwidth]{figures/3J_2T_noSyst_Plus_etalj_MuStack.png}}
%	    \subfigure[]{
%\includegraphics[angle=0,width=0.48\textwidth]{figures/3J_2T_noSyst_Plus_etalj_EleStack.png}}\\
%\subfigure[]{
%\includegraphics[angle=0,width=0.48\textwidth]{figures/3J_2T_noSyst_Minus_etalj_MuStack.png}}
%	    \subfigure[]{
%\includegraphics[angle=0,width=0.48\textwidth]{figures/3J_2T_noSyst_Minus_etalj_EleStack.png}}\\
%\end{center}
%\caption{\label{fig:3JNTeta_pm} light jet $\eta$ in the 3-jets and 1(a-d) or 2(e-h) tags, 
%muons (a,c,e,g) and electrons (b,d,f,h), for positive (a,b,e,f) and negative (c,d,g,h) charge.}
%%(a,c,e runs $\ge$ 166404, \lumiMu \invpb ) and electrons (b,d,f 166404 $\le$  runs $<$ \maxrun, \lumiEleValidation \invpb ) }
%\end{figure}
%
%%Figures ~\ref{fig:3JNTtopMass}(a),(b) show the distribution of the top mass and the peak which is mostly populated by to semi-leptonic events where
%%the b-tagged jet coming from the same top quark as the lepton is chosen, similarly to what happens with single top $t$-channel events.
%\begin{figure}[!h]
%\begin{center}
%
%\subfigure[]{
%\includegraphics[angle=0,width=0.48\textwidth]{figures/3J_1T_noSyst_Plus_topMass_MuStack.png}}
% \subfigure[]{
%\includegraphics[angle=0,width=0.48\textwidth]{figures/3J_1T_noSyst_Plus_topMass_EleStack.png}}\\
%\subfigure[]{
%\includegraphics[angle=0,width=0.48\textwidth]{figures/3J_1T_noSyst_Minus_topMass_MuStack.png}}
% \subfigure[]{
%\includegraphics[angle=0,width=0.48\textwidth]{figures/3J_1T_noSyst_Minus_topMass_EleStack.png}}\\
%\vskip -2ex
%\subfigure[]{
%\includegraphics[angle=0,width=0.48\textwidth]{figures/3J_2T_noSyst_Plus_topMass_MuStack.png}}
% \subfigure[]{
%\includegraphics[angle=0,width=0.48\textwidth]{figures/3J_2T_noSyst_Plus_topMass_EleStack.png}}\\
%\subfigure[]{
%\includegraphics[angle=0,width=0.48\textwidth]{figures/3J_2T_noSyst_Minus_topMass_MuStack.png}}
% \subfigure[]{
%\includegraphics[angle=0,width=0.48\textwidth]{figures/3J_2T_noSyst_Minus_topMass_EleStack.png}}\\
%\end{center}
%\caption{\label{fig:3JNTtopMass_pm} \mt in the 3-jets and 1(a-d) or 2(e-h) tags, muons (a,c,e,g) and electrons (b,d,f,h), for positive (a,b,e,f) and negative (c,d,g,h) charge.}
%%(a,c,e runs $\ge$ 166404, \lumiMu \invpb ) and electrons (b,d,f 166404 $\le$  runs $<$ \maxrun, \lumiEleValidation \invpb ) }
%\end{figure}
%
%%Figure ~\ref{fig:eta3J2TSRSB} shows the agreement in the signal and sideband region for \etalj in the 3-jets, 2-tags sample for the muons and elecrtons.
%
%A reweighting function is extracted from the signal and sideband regions of the 3-jets 2-tags sample. This extraction is performed exactly as in  Refs.~\cite{AN-12-273,CMS-PAS-TOP-12-011}. The extracted function is then applied to the ~\tt~ Monte Carlo distribution in the signal region. 
%Since the ~\tt~ sample is symmetric as a function of the charge, we perform the extraction before separating by charge.
%Figure ~\ref{fig:3J2TetaChargeComparison} shows that, according to simulation, the top model for light jet $\eta$ is independent of the charge of the lepton.
%%This allows to reduce the dependancy on the statistical fluctuations in the 3-jets 2-tags sample.
%
%\begin{figure}[!h]
%\begin{center}
%\subfigure[]{
%\includegraphics[angle=0,width=0.48\textwidth]{figures/ttbar3J2TEtaChargeComparisonMu.png}}
%	    \subfigure[]{
%\includegraphics[angle=0,width=0.48\textwidth]{figures/ttbar3J2TEtaChargeComparisonEle.png}}
%\vskip -2ex
%\end{center}
%\caption{\label{fig:3J2TetaChargeComparison} findme Comparison between the light jet $\eta$ distributions of samples with positive and negative leptons in the $\ttbar$-enriched samples with 3-jets 2-tags sample for muons (a) and electrons (b). }
%\end{figure}
%
%

%
%To cope with any possible modeling effect, we get a reweighting function for the overall $\eta$ distribution
% of the light jet taking the bin-by-bin difference between data and MC. We use this as a systematic uncertainty on the \ttbar\ model for the time being.
%\begin{figure}[!h]
%\begin{center}
%\subfigure[]{
%\includegraphics[angle=0,width=0.48\textwidth]{figures/ttbar/3J_2T_noSyst_etalj_SRMuStack.png}}
%	    \subfigure[]{
%\includegraphics[angle=0,width=0.48\textwidth]{figures/ttbar/3J_2T_noSyst_etalj_SBMuStack.png}}
%\vskip -2ex
%\end{center}
%\vskip -2ex
%\caption{\label{fig:eta3J2TSRSB} Distribution of $|\eta|$ of the jet with the lowest value of b-discriminator
% in the $\ttbar$-enriched samples with 3-jets 2-tags sample inside(a) and outside(b) the 
%$130<\topMass<220$ region. }
%\end{figure}
%
%An extraction procedure is applied in a similar way to what is done in ~\ref{}
%
%
%
%
%
%To extract a reweighting function in the 2-jets 1-tag category, where we intend to perform the signal extraction, we
%first compare in Figure ~\ref{fig:3JNTVS2J1T} shows that the light jet $\eta$ in the 3-jets 1-,and 2-tags categories
%to the one from the 2-jets 1-tag, finding that the 3-jets 2-tag region displays a similar $\eta$ distribution as the one in the 2-jet 1-tag
%sample.
%
%\begin{figure}[!h]
%\begin{center}
%\subfigure[]{
%\includegraphics[angle=0,width=0.48\textwidth]{figures/ttbar/3JNTvs2J1TSR.pdf}}
%	    \subfigure[]{
%\includegraphics[angle=0,width=0.48\textwidth]{figures/ttbar/3JNTvs2J1TSB.pdf}}
%\vskip -2ex
%\end{center}
%\vskip -2ex
%\caption{\label{fig:3JNTVS2J1T} comparison of the MC distribution of $|\eta|$ of the jet with the lowest value of b-discriminator
% in the $\ttbar$-enriched samples with 3-jets N(=1,2) tags  vs the same distribution for the 2-jets 1-tag category for \ttbar\ events inside(a) and outside(b) the 
%$130<\topMass<220$ region. }
%\end{figure}
%
%A reweighting function is extracted, taking the difference in shape between data and expectation, and it is shown 
%in Figure ~\ref{fig:ReweightingFunction} in the Signal(a) and Sideband(b) regions, with the corresponding statistical 
%uncertainties due to data. This does not take into account the MC statistics, which in particular for the signal 
%is crucial in the high $|\eta|$ region of this sample.
%However, such function is therefore applied to the MC distribution in the 2-jets 1-tag region, 
%avoiding the region above $|\etalj| >2.5$ for two reasons: the statistical fluctuations
%in this region significantly affect the shape extraction (see Fig.~\ref{fig:ReweightingFunction})
% and the signal contamination in this region is non negligible (see Fig.~\ref{fig:eta3J2TSRSB}),
%therefore would require a more complicated simultaneous fit to be dealt with effectively.
%
%
%\begin{figure}[!h]
%\begin{center}
%\subfigure[]{
%\includegraphics[angle=0,width=0.48\textwidth]{figures/ttbar/RemodelSR.png}}
%	    \subfigure[]{
%\includegraphics[angle=0,width=0.48\textwidth]{figures/ttbar/RemodelSB.png}}
%\vskip -2ex
%\end{center}
%\vskip -2ex
%\caption{\label{fig:ReweightingFunction} comparison of the MC distribution of $|\eta|$ of the jet with the lowest value of b-discriminator
% in the $\ttbar$-enriched samples with 3-jets N(=1,2) tags  vs the same distribution for the 2-jets 1-tag category for \ttbar\ events inside(a) and outside(b) the 
%$130<\topMass<220$ region. }
%\end{figure}

%Figure ~\ref{fig:3JNTPUZoomMass} shows how the light jet $\eta$ distributions are modified applying a cut in the reconstructed top mass $130 < \topMass < 220$,
%focusing on the $|\eta|>2 $ region.
%From those plots finds out that actually there is an excess of data with respect to MC in the top mass window (a,b), while there is a deficit in the top mass sideband (c,d).
%
%
%\end{figure}
%\begin{figure}[!h]
%\begin{center}
%	    \subfigure[]{
%\includegraphics[angle=0,width=0.48\textwidth]{figures/ttbar/3J_1TtopMassNoPU22TAfterCutsMuStack.png}}
%	    \subfigure[]{
%\includegraphics[angle=0,width=0.48\textwidth]{figures/ttbar/3J_2TtopMassNoPU22TAfterCutsMuStack.png}}
%\vskip -2ex	   
%  \subfigure[]{
%\includegraphics[angle=0,width=0.48\textwidth]{figures/ttbar/3J_1TtopMassNoPU22TAfterCutsMuStack.png}}
%	    \subfigure[]{
%\includegraphics[angle=0,width=0.48\textwidth]{figures/ttbar/3J_2TtopMassNoPU22TAfterCutsMuStack.png}}
%\end{center}
%\vskip -2ex
%\caption{\label{fig:3JNTPUZoomMass} distribution of the light jet $\eta$ in the window $130<\topMass<220$ (a,b) 
%and outside (c,d) in the $\ttbar$-enriched samples with 3-jets and 1(a,c) or 2(b,d) tags. }
%\end{figure}
%
%
%The ratio of events in the signal and sideband region depends on pile up , as it is shown in table ~\ref{tab:SRSBVSPU}. 
%
%Increasing the cut on the $RMS$ of the forward jet reduces the data-mc discrepancy, 
%but does not solve the problem entirely. 
%Table ~\ref{tab:SRSBVSRMS} shows the ratio of events inside and outside the top mass region
%$130 < \topMass < 220$ as function of the $RMS$ of the forward jet.
%
%%//Figure ~\ref{fig:3J2T}

%Where the shape for the qcd variables is extracted from the anti-isolated sample. 
%Table~\ref{tab:tt_control_KSTests} reportes the KS $p$-values for the data-simulation
%comparison of the $\costhetalj$, $\etalj$, and $\topMass$ distributions. 
%Fig.~\ref{fig:tt_control_topMass_TChan_vs_TTBar} shows that the $\topMass$ 
%distribution is similiar between $\ttbar$ and signal for the
%cases where the correct b-tagged jet is taken for reconstruction of the top.

% \begin{table}
% \begin{center}
%   \begin{tabular}{ |l|c|c| }
%  \hline
% Process/Observable  & KS(shape only) data-mc: muon & electron   \\
% \hline
% \etalj            & FIXME & FIXME\\
% \topMass          & 0.85 & 0.90  \\
% \costhetalj       & FIXME & FIXME\\
% \hline
% \end{tabular}
%   \end{center}
% \caption{ KS tests comparing the shapes of the \topMass }%discriminating variables from different samples}
% \label{tab:tt_control_KSTests}
%   \end{table}

%\begin{figure}[!h]
%\begin{center}
%	    \subfigure[]{
%\includegraphics[angle=0,width=0.48\textwidth]{figures/topMass_WSample_noSyst_MuStack.pdf}
%}
%\end{center}
%\vskip -2ex
%\caption{\label{fig:tt_control_topMass_TChan_vs_TTBar} Distribution of $\topMass$ in the $\ttbar$-enriched sample 
%for $t$-channel signal and $\ttbar$ events, requiring that the b jet used to reconstruct
%$\topMass$ is the one from top decay. Left: muons, right: electrons. FIXME}
%\end{figure}


%The shape extraction procedure is the following: $W+$light shape is obtained subtracting the contributions of all other channels from data in the following way: QCD shapes are taken from anti-isolated samples, qcd yields are taken from the fit to mtw and extrapolated after the cut to $M_T$, all other channels are taken from Monte Carlo.
