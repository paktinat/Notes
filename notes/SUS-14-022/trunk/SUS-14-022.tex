%\usepackage{tcolorbox}
% Customizable fields and text areas start with % >> below.
% Lines starting with the comment character (%) are normally removed before release outside the collaboration, but not those comments ending lines

% svn info. These are modified by svn at checkout time.
% The last version of these macros found before the maketitle will be the one on the front page,
% so only the main file is tracked.
% Do not edit by hand!
\RCS$Revision: 302527 $
\RCS$HeadURL: svn+ssh://svn.cern.ch/reps/tdr2/notes/SUS-14-022/trunk/SUS-14-022.tex $
\RCS$Id: SUS-14-022.tex 302527 2015-09-03 20:04:06Z claudioc $
%%%%%%%%%%%%% local definitions %%%%%%%%%%%%%%%%%%%%%
% This allows for switching between one column and two column (cms@external) layouts
% The widths should  be modified for your particular figures. You'll need additional copies if you have more than one standard figure size.
\newcommand{\mt}{\ensuremath{\,{m_{\rm T}}}\xspace}
\newcommand{\mttwo}{\ensuremath{\,{M_{\rm T2}}}\xspace}
\newcommand{\mindphifour}{\ensuremath{\,{\Delta\phi_{4}^{\rm min}}}\xspace}
\newcommand{\IL}{\ensuremath{19.6\,\rm{fb^{-1}}}\xspace}
\newcommand{\SumMT}{ \ensuremath{\,\Sigma m_{\rm T}^{\tau_i}}\xspace}
\newcommand{\hadtau}{\ensuremath{\tau_{\rm had}}\xspace}
\newcommand{\Tau}{\ensuremath{\tau_{\rm had}}\xspace}
\newcommand{\tauMT}{\ensuremath{\,m_{\rm T}^{\hadtau}}\xspace}
\newcommand{\tauTau}{\ensuremath{\,\hadtau\hadtau}\xspace}
\newcommand{\muTau}{\ensuremath{\,\mu\hadtau}\xspace}
\newcommand{\leptonTau}{\ensuremath{\,l\hadtau}\xspace}
\newcommand{\chione}{\ensuremath{\widetilde{\chi}^{\pm}_1}\xspace}
\newcommand{\mvisi}{\ensuremath{m^{{\rm vis}(i)}}\xspace}
\newcommand{\etvisi}{\ensuremath{\et^{{\rm vis}(i)}}\xspace}
\newcommand{\vptvisi}{\ensuremath{\ptvec^{{\rm vis}(i)}}\xspace}
\newcommand{\PTslashvec}{\ensuremath{{\displaystyle{\not}\vec{p}}_{\rm T}}\xspace}
\newcommand{\ptvecmiss}{\ensuremath{\ptvec^{\rm miss}}\xspace}
\newcommand{\PSGcmDo}{\ensuremath{\widetilde{\chi}_{1}^{-}}\xspace}
\newcommand{\PSGcpDo}{\ensuremath{\widetilde{\chi}_{1}^{+}}\xspace}
\newcommand{\binone}{bin \uppercase\expandafter{\romannumeral 1}\,}
\newcommand{\bintwo}{bin \uppercase\expandafter{\romannumeral 2}\,}


\newlength\cmsFigWidth
\ifthenelse{\boolean{cms@external}}{\setlength\cmsFigWidth{0.85\columnwidth}}{\setlength\cmsFigWidth{0.4\textwidth}}
\ifthenelse{\boolean{cms@external}}{\providecommand{\cmsLeft}{top\xspace}}{\providecommand{\cmsLeft}{left\xspace}}
\ifthenelse{\boolean{cms@external}}{\providecommand{\cmsRight}{bottom\xspace}}{\providecommand{\cmsRight}{right\xspace}}
%%%%%%%%%%%%%%%  Title page %%%%%%%%%%%%%%%%%%%%%%%%
\cmsNoteHeader{SUS-14-022} % This is over-written in the CMS environment: useful as preprint no. for export versions
% >> Title: please make sure that the non-TeX equivalent is in PDFTitle below
\title{Search for electroweak production of charginos in final states with two tau leptons in pp collisions at $\sqrt{s}= 8$\TeV}

% >> Authors
%Author is always "The CMS Collaboration" for PAS and papers, so author, etc, below will be ignored in those cases
%For multiple affiliations, create an address entry for the combination
%To mark authors as primary, use the \author* form
\address[neu]{Northeastern University}
\address[fnal]{Fermilab}
\address[cern]{CERN}
\author[cern]{The CMS Collaboration}


% >> Date
% The date is in yyyy/mm/dd format. Today has been
% redefined to match, but if the date needs to be fixed, please write it in this fashion.
% For papers and PAS, \today is taken as the date the head file (this one) was last modified according to svn: see the RCS Id string above.
% For the final version it is best to "touch" the head file to make sure it has the latest date.
\date{\today}

% >> Abstract
% Abstract processing:
% 1. **DO NOT use \include or \input** to include the abstract: our abstract extractor will not search through other files than this one.
% 2. **DO NOT use %**                  to comment out sections of the abstract: the extractor will still grab those lines (and they won't be comments any longer!).
% 3. For PASs: **DO NOT use tex macros**         in the abstract: CDS MathJax processor used on the abstract doesn't understand them _and_ will only look within $$. The abstracts for papers are hand formatted so macros are okay.
\abstract{
 A search for electroweak production of supersymmetric particles is performed 
with two tau leptons in the final state.
 These results are based on 18.1 to 19.6  fb$^{-1}$  of proton-proton collisions at $\sqrt{s}$ = 8\TeV, 
 collected with the CMS detector at the CERN Large Hadron Collider.
The observed events are found to be consistent with the standard model prediction. Upper limits are set on the masses of the lightest 
chargino and the lightest neutralino, assuming the third generation sleptons are the lightest sleptons and their masses are at a middle 
point between the chargino and the neutralino. In the context of simplified model spectra, charginos lighter than 417\GeV are excluded 
at 95\% confidence level in the case of massless neutralino.
}    

% >> PDF Metadata
% Do not comment out the following hypersetup lines (metadata). They will disappear in NODRAFT mode and are needed by CDS.
% Also: make sure that the values of the metadata items are sensible and are in plain text:
% (1) no TeX! -- for \sqrt{s} use sqrt(s) -- this will show with extra quote marks in the draft version but is okay).
% (2) no %.
% (3) No curly braces {}.
\hypersetup{%
pdfauthor={IPM TauTau Group},%
pdftitle={Search for electroweak production of charginos in di-tau final states in pp collisions at sqrt(s)= 8 TeV with the CMS detector},%
pdfsubject={CMS},%
pdfkeywords={CMS, physics, software, computing}}

\maketitle %maketitle comes after all the front information has been supplied
% >> Text
%%%%%%%%%%%%%%%%%%%%%%%%%%%%%%%%  Begin text %%%%%%%%%%%%%%%%%%%%%%%%%%%%%
%% **DO NOT REMOVE THE BIBLIOGRAPHY** which is located before the appendix.
%% You can take the text between here and the bibiliography as an example which you should replace with the actual text of your document.
%% If you include other TeX files, be sure to use "\input{filename}" rather than "\input filename".
%% The latter works for you, but our parser looks for the braces and will break when uploading the document.
%%%%%%%%%%%%%%%
\section{Introduction}
\label{sect:introduction}

Supersymmetry \cite{Martin:1997ns} (SUSY) is one of the most promising extensions of the 
Standard Model of the elementary particles (SM) which solves both the 
quadratic divergencies and hierarchy problems simultaneously. It introduces a new symmetry between the bosons and fermions and 
for every particle a sparticle is defined which is exactly the same, but differ in spin by 1/2. 
%Since the super particles are not discovered yet, the supersymmetry should be a broken symmetry. 
%Various mechanisms are introduced to break the symmetry softly without changing the other interesting features of the theory.

In a hadron collider, like LHC, it is expected to see the signature of the coloured SUSY partners, 
but the very extensive search in the LHC experiments pushes the mass of the coloured particles much 
beyond the previous expectations. 
Looking at other sectors of the SUSY, e.g, electroweak production of the sparticles, is motivated not to miss SUSY in a corner.
A search for new physics using 20 $\fbinv$ of data from CMS taken in 2012 is documented in this note. 
Although the search is sensitive to any high scale 
new physics with a missing transverse momentum, an R-parity conserving SUSY model is used 
to illustrate the performance of the method.

Due to the special role of the third generation of the sparticles, events with two taus in the final state 
accompanying with the missing transverse energy (\MET) are considered.
The two taus can be generated in the cascade of the staus or charginos:
\begin{linenomath}
\begin{equation}
p + p \rightarrow \PSGcpDo +\PSGcmDo ~~\mathrm{or}~~  p + p \rightarrow \sTau + \sTau
\end{equation}
\end{linenomath}
when 
\begin{linenomath}
\begin{equation}
\PSGcpDo \rightarrow \sTau + \nu ~~\mathrm{or}~~  \PSGcpDo \rightarrow \sNu_{\tau} + \tau 
\end{equation}
\end{linenomath}
and 
\begin{linenomath}
\begin{equation}
\sTau \rightarrow \tau + \PSGczDo ~~\mathrm{or}~~  \sNu_{\tau} \rightarrow \nu + \PSGczDo 
\end{equation}
\end{linenomath}
and $\PSGczDo$ can not be detected and appears as missing transverse momentum (\MET).
In this note, we mainly focus on the $\PSGcpDo\PSGcmDo$ production which has a higher 
production cross section. Figure \ref{fig:Productions} shows our favourite decays.

\begin{figure}[!htb]
\centering
\includegraphics[width=0.49\textwidth]{Introductionfigs/DiChargino.png}
\includegraphics[width=0.49\textwidth]{Introductionfigs/DiSTau.png}
\caption{Schematic production of double tau from chargino pair and stau pair.}
\label{fig:Productions}
\end{figure}

In the context of MSSM, supersymmetric objects are produced in pairs to conserve R-parity quantum number. Therefore in the production of charginos from proton-proton collisions, one may consider the following interaction $$pp\rightarrow\PSGcpDo\PSGcmDo\rightarrow\sTau^{+}\nu_\tau \sTau^{-}\nu_\tau\rightarrow\tau^{+}\PSGczDo\nu_\tau\tau^{-}\PSGczDo\nu_\tau\rightarrow\tau^{+}\tau^{-} + 2\,\PSGczDo + 2\,\nu_\tau.$$ In the detector, what can be observed are only 2 tau leptons plus missing transverse energy from the presence of neturinos and neutralinos in the final state.\\ 
The tau leptons can decay leptonically, in 35\% of the time, or can decay via hadrons, which is occurring with 65\% of probability. Since there are two leptons in the final state, then the probability of having $\PSGcpDo\PSGcmDo \rightarrow \hadtau+e/\mu+\MET$ is about 46\%, while the probability of occurring $\PSGcpDo\PSGcmDo \rightarrow \hadtau+\hadtau+\MET$ is about 42\%. Hereafter, final states containing a lepton are referred to as $\leptonTau$ channel and those events where the two taus decays via hadrons are referred to as $\tauTau$. The selction cuts to enhance $\tauTau$ events will be discussed in Section~\ref{sect:tauTauCuts}. The list of cuts to select $\leptonTau$ events can be found in Section~\ref{sect:leptonTauCuts}.\\     

The search variable is the stransverse mass (\mttwo) which is the natural extension of the known transverse mass (\mt) to a case 
when two massive particles with equal mass are created in pairs and decay via a chain of jets and leptons to two 
invisible particles. 
In the case of R-Parity conserving SUSY, the Lightest Supersymmetric Particle (LSP) escapes the detection and appears as 
a missing transverse momentum.
The distribution of \mttwo reflects the scale of the produced particles and is much higher for sparticles
compared to the SM particles. Hence, SUSY should appear as an excess in the tail of the \mttwo distribution.
It was shown previously \cite{MT2_2011} that \mttwo is a powerful variable to search for SUSY. 


After introduction in the next section the \mttwo variable is introduced. 





\section{The CMS detector and event reconstruction}
\label{sect:CMSRec}
The central feature of the CMS apparatus is a superconducting solenoid of 6\unit{m} internal diameter, providing a magnetic field of 3.8\unit{T}. Within the solenoid volume are a silicon pixel and strip tracker, a lead tungstate crystal electromagnetic calorimeter, and a brass and scintillator hadron calorimeter, each composed of a barrel and two endcap sections. Muons are measured in gas-ionization detectors embedded in the steel flux-return yoke outside the solenoid. Extensive forward calorimetry complements the coverage provided by the barrel and endcap detectors. 
A more detailed description of the CMS detector, together with a definition of the coordinate system used and the relevant kinematic variables, can be found in Ref. \cite{Chatrchyan:2008zzk}.
%For completeness, a short review is given here. 


Events from pp interactions must satisfy the requirements of a two-level trigger system.
The first level of the CMS trigger system, composed of custom hardware processors, uses information from the calorimeters and muon detectors to select the most interesting events in a fixed time interval of less than 4\mus. The high-level trigger processor farm further decreases the event rate from around 100\unit{kHz} to less than 1\unit{kHz}, before data storage. 

The particle-flow (PF) algorithm~\cite{CMS-PAS-PFT-09-001,CMS-PAS-PFT-10-001} reconstructs and identifies each individual particle with an optimized combination of information from the various elements of the CMS detector. 
%The energy of photons is directly obtained from the ECAL measurement, corrected for zero-suppression effects. The energy of electrons is determined from a combination of the electron momentum at the primary interaction vertex as determined by the tracker, the energy of the corresponding ECAL cluster, and the energy sum of all bremsstrahlung photons spatially compatible with originating from the electron track. The energy of muons is obtained from the curvature of the corresponding track. The energy of charged hadrons is determined from a combination of their momentum measured in the tracker and the matching ECAL and HCAL energy deposits, corrected for zero-suppression effects and for the response function of the calorimeters to hadronic showers. Finally, the energy of neutral hadrons is obtained from the corresponding corrected ECAL and HCAL energy. 
Jets are reconstructed from the PF candidates with the anti-$k_t$ clustering
algorithm~\cite{Cacciari:2008gp} with a distance parameter of 0.5. We apply
 (transverse momentum) \pt- and (pseudorapidity) $\eta$-dependent corrections to account for residual
effects of nonuniform detector response~\cite{Chatrchyan:2011ds}.
A correction to account for multiple pp collisions within the same or nearby
bunch crossings (pileup interactions) is estimated on an event-by-event basis using the
jet area method described in Ref.~\cite{Cacciari:2007fd}, and is
applied to the reconstructed jet \pt.
The combined secondary vertex algorithm is used to identify (``b tag'') jets 
originating from b quarks.  This algorithm 
 is based on the reconstruction of secondary vertices, together with track-based lifetime information~\cite{Chatrchyan:2012jua}. 
%In this analysis the "medium" working point is used. 
%The working point corresponds to an average b-tagged jets efficiency of 70\%, 
In this analysis a working point is chosen such that, for jets with a \PT value greater than 60\GeV, the efficiency for tagging a jet containing a b quark is 50\%, with a light-parton jet misidentification rate of 1.5\%, and $\cPqc$ quark jet misidentification rate of 20\%.
Scale factors are applied to the simulated events to reproduce the tagging efficiencies measured in data, 
separately for jets originating from b or $\cPqc$ quarks and from light-flavor partons.
Jets with  \PT $>$ 40\GeV and $\abs{\eta} < 5.0$ and b-tagged jets with \PT $>$ 20\GeV and $\abs{\eta} < 2.4$ are considered in this analysis.


The PF candidates are used to reconstruct the missing transverse momentum  vector \ptvecmiss, defined as the negative of the vector sum of the transverse momenta of all reconstructed particles.  
%Corrections are applied to ensure consistency between
%\ptvecmiss and the corrections to the jet energies described above.  
%The missing transverse momentum in the event (\MPT) is defined as the magnitude of \ptvecmiss.
In the event \MPT is defined as the magnitude of \ptvecmiss.

Hadronically decaying $\tau$ leptons are reconstructed using the hadron-plus-strips algorithm~\cite{Khachatryan:2015dfa}.
The constituents of the reconstructed jets are used to identify individual $\tau$ lepton decay modes with one charged 
hadron and up to two neutral pions, or three charged hadrons. 
Additional discriminators are used to separate \Tau from electrons and muons.
Prompt $\tau$ leptons are expected to be isolated in the detector.
To discriminate them from Quantum ChromoDynamics (QCD) jets, we use a measure of isolation 
based on the transverse momentum of charged hadrons and photons falling within 
a cone around the $\tau$ lepton momentum direction after correcting for the effect of
pileup \cite{Khachatryan:2014wca}. The ``loose'', ``medium'', and ``tight'' working points are defined
by requiring the measure of isolation not to exceed thresholds of 2.0, 1.0,
and 0.8 GeV, respectively.
 A similar isolation algorithm is 
used in this analysis to separate leptons (e or $\mu$) from $\tau$ lepton decays from 
those arising from hadron decays within jets.

\section{The Monte Carlo samples}
\label{sect:MCSamples}
Events from SM processes that may represent significant sources of backgrounds, $\cPZ$+jets, \wjets, $\cPqt\cPaqt$ and diboson, 
are generated using the \MADGRAPH 5.1~\cite{Alwall:2011uj} generator. 
Single top quark and Higgs boson events are generated with {\POWHEG} 1.0~\cite{Nason:2004rx,Frixione:2007vw,Alioli:2009je,Alioli:2010xd}.
In the following, the events containing at least one top quark or one $\cPZ$ boson are referred to as ``Top'' and ``ZX'', respectively. 
Events from Higgs boson production via gluon fusion, vector boson fusion, or in association with a $\cPZ$ or $\PW$ boson, or a \ttbar pair are referred to as ``Higgs''. The masses of the top quark and Higgs boson are set to be 172.5\GeV and 125\GeV, respectively.

%The simplified model which is used to describe the signal events is shown in Fig.~\ref{fig:Productions} (left). 
In signal samples, pairs of charginos (\chione) 
are produced and decay exclusively to the final states that contain two $\tau$ leptons, two $\tau$ neutrinos, and %$\tau$, two $\nu_{\tau}$ and 
two neutralinos, (\PSGczDo) as shown in Fig.~\ref{fig:Productions} (left). 
The mediator in the decay of the \chione can be either a \sTau or $\sNu_{\tau}$. 
The masses of the \sTau and $\sNu_{\tau}$ are set to be equal to the mean value of the \chione and \PSGczDo masses. 
Thus they are produced on-shell.
The two distinct decay chains in the left diagram of Fig.~\ref{fig:Productions} 
are assumed to have equal branching fractions of 50\%. 
For description of the parton shower and fragmentation, all generators are interfaced with \PYTHIA 6.426~\cite{Sjostrand:2006za}, 
%\PYTHIA
which is also used to generate signal events (chargino pair production). To improve the modeling of the $\tau$ lepton decays, 
we use the \TAUOLA 1.1.1a~\cite{Davidson:2010rw} package. 


In the data set considered in this paper,
there are on average 21 proton-proton interactions in each bunch crossing.
Additional interactions are generated with \PYTHIA and superimposed on simulated events in a manner consistent with 
 instantaneous luminosity profile of the data set.
The detector response in the  Monte Carlo (MC) background event samples is modeled by a
detailed simulation
of the CMS detector based on {\GEANTfour}~\cite{Agostinelli:2002hh}.  
To reduce  computational requirements, signal events 
are processed by the CMS fast simulation \cite{Abdullin:2011zz} instead of {\GEANTfour}. 
It is verified that the CMS fast simulation gives a reasonable agreement with the detailed simulation for our favorite signal which has hadronic decays of 
tau leptons in the final state.
All simulated events are reconstructed with the same algorithms used for collision data.

The SM backgrounds are normalized using the most accurate calculations of the cross sections available 
in the literature. These cross sections correspond to next-to-next-to-leading-order (NNLO) accuracy for $\cPZ$+jets~\cite{Melnikov:2006kv} 
and \wjets~\cite{xsec_WZ} events. For the $\cPqt\cPaqt$ simulated samples, the cross section used is calculated to full NNLO accuracy including
%The cross section of $\cPqt\cPaqt$ simulated sample at full NNLO accuracy including 
the resummation of next-to-next-to-leading-logarithmic (NNLL) terms~\cite{Czakon:2011xx}. %is used~\cite{Czakon:2011xx}. 
The event yields from diboson production are normalized to the next-to-leading-order (NLO) cross section  taken from Ref.~\cite{Campbell:2011bn}. 
The \textsc{Resummino}~\cite{Fuks:2012qx,Fuks:2013vua,Fuks:2013lya} calculations at NLO+NLL accuracy are used to calculate the signal cross sections, where 
NLL refers to next-to-leading-logarithmic precision.

\section{\texorpdfstring{The definition of $\rm {\mttwo}$}{The definition of MT2}}
\label{sect:mt2def}
The $\mttwo$ variable~\cite{Lester:1999tx,Barr:2003rg} is used in this analysis to discriminate between the SUSY signal and the SM backgrounds as proposed in~\cite{Barr:2009wu}. The variable is originally introduced to measure the mass of primary pair-produced particles, decaying eventually to undetected particles (e.g. neutralino). Assuming the two primary supersymmetric particles undergo the same decay chain with visible and undetectable particles in the final state, the system can be described by the visible mass ($\mvisi$), transverse energy ($\etvisi$) and transverse momentum ($\vptvisi$) of each branch ($i=1,2$) together with missing transverse momentum ($\ptvecmiss$) which is shared between the two. The $\ptvecmiss$ is considered as the sum of the neutralinos transverse momenta, $\vec{p}_{\rm T}^{\PSGczDo(i)}$. For decay chains including SM neutrinos, the $\ptvecmiss$ gets some generally minor contributions from $\pt^{\nu}$'s.

The transverse mass of each branch can be written as 
\begin{linenomath}
\begin{equation}
\label{eq:mtdef}
(\mt^{(i)})^{2}= (\mvisi)^2+m^2_{\PSGczDo}+2(\etvisi\et^{\PSGczDo(i)}-\vptvisi\dot\pt^{\PSGczDo(i)}).
\end{equation}
\end{linenomath}
For the correct neutralino mass, this distribution has an endpoint at the mass of the primary particle, similar to the W boson transverse mass used to measure $m_{\rm W}$~\cite{Arnison:1983rp,Banner:1983jy,Affolder:2000bpa,Abazov:2002bu}. 

As a generalization to the transverse mass, the $\mttwo$ variable is proposed to overcome the problem of unknown $\pt^{\PSGczDo(i)}$. The kinematic endpoint of $\mttwo$ carries model independent information about the mass difference between the primary and the secondary particles. In the current analysis and for a given $m_{\PSGczDo}$, the $\mttwo$ variable is defined as
\begin{linenomath}
\begin{equation}
\label{eq:mt2def}
\mttwo(m_{\PSGczDo})= \min_{\vec{p}_{\rm T}^{\PSGczDo(1)}+\vec{p}_{\rm T}^{\PSGczDo(2)}=\ptvecmiss}\,\left[\,\max\,\{ \, \mt^{(1)},\,\mt^{(2)}\,\}\,\right].
\end{equation}
\end{linenomath}
For the correct value of $m_{\PSGczDo}$, the kinematic endpoint of the $\mttwo$ distribution is at the mass of the primary particle while it shifts accordingly with the $m_{\PSGczDo}$ being lower or higher than the correct value. In this analysis we set $m_{\PSGczDo}=0$. The visible part of the event here changes from two hadronically decaying $\tau$ leptons ($\hadtau$) to a combination of a muon or an electron with a $\hadtau$ candidate. 

Because of our choice for $m_{\PSGczDo}=0$, the resulting $\mttwo$ variable is expected to well reject not only events with no genuine $\MET$ but events with a back-to-back topology ($\mttwo=0$) regardless of the value of $\MET$ or $\pt$. 

\section{\texorpdfstring{Event selection for the \tauTau channel}{Event selection for the tau-tau channel}}
\label{sect:tauTauCuts}
In this channel data of proton-proton collisions,  corresponding to an integrated luminosity of 18.1 $\mathrm{fb}^{-1}$, are used.
The events are first selected with a trigger \cite{Chatrchyan:2011nv} that requires the presence of two isolated 
\Tau candidates with \PT $>$ 35\GeV and $|\eta|<$ 2.1, passing loose identification requirements.
Offline, the two \Tau candidates must pass the tight $\tau$ isolation discriminator,
\PT $>$ 45\GeV and $|\eta|<$ 2.1, and have opposite sign (OS).
In events with more than one \tauTau pair, only the pair with the most isolated \Tau objects is considered. 

Events with extra isolated electrons or muons of \PT $>$ 10\GeV and $|\eta| <$ 2.4 
are rejected to suppress backgrounds from diboson decays. 
Inspired from the MC studies, 
the contribution from the \Z$ \rightarrow$ \tauTau  background is reduced by rejecting events 
where the visible di-\Tau invariant mass is between 55 and 85\GeV (\Z boson veto).  
Furthermore, contributions from low-mass DY and QCD multijet production are 
reduced by requiring the invariant mass to be greater than 15\GeV.
To further reduce \Z $\rightarrow$ \tauTau and QCD multijet events, 
\MPT $>$ 30\GeV and \mttwo $>$ 40\GeV are also required.
The minimum angle \deltaphi in the transverse plane between the \ptvecmiss and any of the \Tau and jets, 
including b-tagged jets, must be greater than 1.0 radians. 
This requirement reduces backgrounds from QCD multijet events and \wjets events.

After applying the preselection described above,
additional requirements are introduced to define two search regions.
The first search region (\binone) targets models with a large mass difference ($\Delta m$) 
between charginos and neutralinos.
In this case, the \mttwo signal distribution can have a long tail beyond the 
distribution of SM backgrounds.
The second search region (\bintwo) is dedicated to models with small values of $\Delta m$.
In this case, the sum of the two transverse mass values, \SumMT = $\mt(\Tau^1,\ptvecmiss) + \mt(\Tau^2,\ptvecmiss)$, 
provides additional discrimination between signal and SM background processes.

The two signal regions (SR) are defined as:
\begin{itemize}
\item {\bf \binone}: \mttwo $>$ 90\GeV;
\item {\bf \bintwo}:  \mttwo $<$ 90\GeV, \SumMT $>$ 250\GeV, and b-tagged jets are vetoed.
\end{itemize}
The veto on b-tagged jets in SR2 reduces the
number of \ttbar events, which
are expected in  the low-\mttwo region. Table \ref{Tab.Cuts} summarizes the selection requirements for the different signal regions.

\section{\texorpdfstring{Event selection for the \leptonTau channel}{Event selection for the lepton-tau channel}}
\label{sect:eleTauCuts}
Events in the \leptonTau final states (e\Tau and $\mu\Tau$)
are collected with triggers that require 
a loosely isolated \Tau with \PT $>$ 20\GeV and $|\eta|$ $<$ 2.3, as well as
an isolated electron or muon with $|\eta| < 2.1$ \cite{Chatrchyan:2011nv,Khachatryan:2015hwa,Chatrchyan:2012xi}.  The minimum
\PT requirement for the electron (muon) was increased during the data taking from 20 to 22\GeV (17 to 18\GeV)
due to the increase in instantaneous luminosity. An integrated luminosity of 19.6 $\mathrm{fb}^{-1}$ is used to study these channels.

In the offline analysis, the electron, muon, and \Tau objects are required to have \PT $>$ 25, 20, and 25\GeV, respectively, 
and the corresponding identification and isolation requirements are tightened. The $|\eta|$ requirements are same as those in the online selections.
In events with more than one opposite-sign \leptonTau pair, we only consider
 the pair that maximizes the scalar $\pt$ sum of \Tau and electron or muon.  Events with additional loosely isolated leptons
with \PT $>$ 10\GeV are rejected to suppress backgrounds from $Z$ boson
decays.  

Just as for the \tauTau channel, we apply preselection requirements to suppress
QCD multijet, \ttbar, $Z \to \tau \tau$, and low-mass resonance events.
These requirements are: \mttwo $>$ 40\GeV, \MPT $>$ 30\GeV, \leptonTau 
invariant mass between 15 and 45\GeV or $>$ 75\GeV, \deltaphi $>$ 1.0 radians. We also reject events with b-tagged jets to reduce the 
\ttbar background.
 The final signal region requirements are \mttwo $>$ 90\GeV and the sum of the transverse mass of the two final leptons
(\tauMT $>$ 200\GeV). %where \tauMT is the \Tau transverse mass 
The latter requirement provides discrimination against the \wjets background.  Unlike in the \tauTau channel,
events with \mttwo $<$ 90\GeV are not used because of the higher 
level of background. 

The summary of the selection requirements is shown in Table \ref{Tab.Cuts}.
Figure \ref{fig:mt2leptontau} % and \ref{fig:taumtleptontau} 
shows the \mttwo distribution after the preselection requirements are imposed. 
%and the \tauMT distribution after the preselection and the \mttwo requirements, respectively.
The data are in good agreement with the SM expectations, evaluated from MC simualtion, within the statistical uncertainties. 
A SUSY signal corresponding to a high mass difference 
 $(m_{\chione}=380\GeV,~m_{\PSGczDo}=1\GeV)$ is used to show the expected signal distribution.

\begin{figure}[!htb]
\centering
\includegraphics[angle=0,scale=0.375]{SelectionEleTau/MT2_eletau.pdf}
\includegraphics[angle=0,scale=0.375]{SelectionMuTau/MT2_mutau.pdf}
\caption{\mttwo  distributions for events in the sample after preselection, compared to SM expectation in (left) e$\tau_{h}$ and (right) \muTau channels. The signal distribution is shown for $m_{\chione}=380\GeV,~m_{\PSGczDo}=1\GeV$. The last bins include the higher values of \mttwo also. Only the statistical uncertainties are shown.}
\label{fig:mt2leptontau}
\end{figure}







\section{Background Estimation}
\subsection{\texorpdfstring{QCD Background Estimation in $\tauTau$ Channel}{QCD Background Estimation in tau-tau Channel}}
\label{sect:bkg}

%In this section, data driven methods are applied to estimate the contribution of
 %  the main backgrounds in the signal region.

In QCD multi-jet events all tau candidates are misidentified as jets. Due to large cross
section and
the poor MC modeling of the tau misidentification rate from jets, the QCD multi-jet contribution in the signal regions is estimated from data using the ABCD" method.

This method indeed relies on different distributions of QCD
in the four exclusive regions labelled as A, B, C (the control regions) and D (the signal region) are defined in a two-dimensional plane as a function of uncorrelated discriminating variables.
In this case the number of QCD events in signal region D can be calculated from the number of QCD events in the control region A multiplied in the ratio of the number of QCD events in the control region C to QCD events in control region B$(T=C/B)$.Figure~\ref{fig:1QCDbg} 

\begin{figure}[htbp]
\centering
\includegraphics[width=0.49\textwidth]{QCDbginTauTau/Bin1_transferfactor.png}
\includegraphics[width=0.49\textwidth]{QCDbginTauTau/Bin2_transferfactor.png} \\
\caption{The ratio of the number of QCD events in the control region C to QCD events in control region B.The
fit line  is drawn in red.
 Left:  $\mttwo>90$ Bin1   Right:  $\SumMT >250$ Bin2.}
\label{fig:1QCDbg}
\end{figure}

The tau identification criterion (tau-id) and a kinematic variable chosen depending (\mttwo in Bin 1 and \SumMT in Bin2) 
on the SR are used as the two uncorrelated discriminating variables to define the regions A, B, C and D. The definitions of the control regions are summarized in table \ref{2QCDbg}.

\begin{table}
\begin{center}
\begin{tabular}{|c|c|c|c|}
\hline
Region&A& B & C
\\ \hline\hline
$\mttwo>90$ Bin1 &$\mttwo >90$ & $\mttwo <90$&$\mttwo <90$ \\
 &at least 1 loose taus&at least 1 loose taus& loose tau veto\\
 &loose-loose loose-medium &loose-loose loose-medium &medium-medium \\
 &loose-tight&loose-tight&medium-tight tight-tight\\ 
 &No cut on charge&No cut on charge& Sum charge==0\\
\hline
$\SumMT>250$ Bin2 &$\SumMT >250$ &$\SumMT <250$&$\SumMT < 250$\\
 &at least 1 loose taus&at least 1 loose taus& loose tau veto\\
 &loose-loose loose-medium &loose-loose loose-medium &medium-medium \\
 &loose-tight&loose-tight&medium-tight tight-tight\\
 &No cut on charge&No cut on charge& Sum charge==0\\
% &misc.MinMetJetDphiPt40$>$1 is relaxed\\

\hline
\end{tabular}
\caption{The requirement on the kinematic variables used to define the control regions A,B,C.
$\mindphifour>1$ cut is relaxed. }
\label{2QCDbg}
\end{center}
\end{table}

The number of QCD multi-jet events in the control regions is estimated from data after subtraction of other SM contributions estimated from MC simulation.

In order to increase the data statistics, the cut on the $\mindphifour>1$ is relaxed.This cut was
introduced to suppress the QCD background events,now that we want to estimate QCD multi-jet background this cut is relaxed  .The only the ratio this efficiency should be
applied into account QCD in the control regions to estimate the number of QCD events in the signal region.

The fraction of QCD events with all selection cuts with respect to the QCD events with all selection cuts but the
$\mindphifour>1$ are shown in Figure~\ref{fig:3QCDbg} .

\begin{figure}[htbp]
\centering
\includegraphics[width=0.49\textwidth]{QCDbginTauTau/Bin1_miscefficiency.png}
\includegraphics[width=0.49\textwidth]{QCDbginTauTau/Bin2_miscefficiency.png} \\
\caption{ Ratio between QCD multi-jet events passing all selection cuts versus QCD events
 passing all selection cuts but $MinMetJetDphiPt40>1$. Left:  $\mttwo>90$ Bin1   Right:  $\SumMT >250$ Bin2.}
\label{fig:3QCDbg}
\end{figure}




The results of the ABCD method are summarized in table \ref{4QCDbg} and the distributions of the kinematic variables  for data, SM backgrounds and SUSY are shown in the Figure~\ref{fig:5QCDbg}. The SM background distributions
are taken from MC simulation, except for the QCD-multi-jet contribution, which is estimated
using the ABCD method.



\begin{table}
\begin{center}
\scalebox{0.92}{
\begin{tabular}{|l|c|c|c|c|c|c|c|}
\hline
 & Sample & RegionA & RegionB & RegionC & T=C/B & QCD in Signal
 Region(D)\\
\hline\hline
\multirow{7}{*}{\mttwo$>$90}& Data&10.00 +- 3.16 & 880.00 +- 29.66&  430.00 +- 20.74& \multirow{7}{*}{0.43+-0.32} & \multirow{7}{*}{0.06 +- 0.08}\\ \cline{2-5}

&Z+jets& 2.27 +- 0.90 &29.27 +- 3.22 & 51.45 +- 4.43 & & \\\cline{2-5}

&W+jets& 2.90 +- 1.43&69.20 +- 8.70 &49.25 +- 7.22 & & \\\cline{2-5}

&WW+jets&0.12 +- 0.07 &0.76 +- 0.17 &1.60 +- 0.25 & & \\\cline{2-5}

&Top& 0.49 +- 0.47&21.78 +- 3.13 & 14.51 +- 2.63& & \\\cline{2-5}
&QCD& 4.23 +- 3.62 & 758.99+-31.24& 313.19+-22.55& & \\\cline{2-5}
&Susy& 1.46 +- 0.17& 3.96 +- 0.28& 9.01 +- 0.41& & \\\cline{2-5}
\hline\hline\hline
\multirow{7}{*}{$\SumMT>250$}&Data  &25.00 +- 5.00  &723.00 +- 26.89 &348.00 +- 18.65 & \multirow{7}{*}{ 0.41 +- 0.03} & \multirow{7}{*}{0.61 +- 1.55} &  \\
\cline{2-5}


&Z+jets& 2.22 +- 1.07 & 21.78 +- 2.72 & 39.57 +- 3.94& & \\\cline{2-5}

&W+jets&  4.28 +- 1.46&51.84 +- 7.48 & 40.09 +- 6.82& & \\\cline{2-5}

&WW+jets&  0.09 +- 0.05& 0.42 +- 0.13& 0.87 +- 0.19 & & \\\cline{2-5}

&Top&0.42 +- 0.41 &3.07 +- 1.22 & 3.31 +- 1.43& & \\\cline{2-5}
&QCD&18.00 +- 5.33 &645.89+-28.07 & 264.16+-20.30& & \\\cline{2-5}
&Susy& 2.13 +- 0.20&1.18 +- 0.15 &  2.80 +- 0.22& & \\\cline{2-5}
\hline\hline

\end{tabular}}
\caption{ The MC predicted backgrounds in the multi-jet control regions, including both the
statistical and systematic uncertainties, and the expected multi-jet contribution,obtained
by subtracting the MC contributions from observed data . Predicted event yields for the
SUSY  in the control regions are also shown. The estimated multi-jet contribution
in the SRs is given in the last column.
}
\label{4QCDbg}
\end{center}
\end{table}


\begin{figure}[htbp]
\centering
\includegraphics[width=0.49\textwidth]{QCDbginTauTau/Bin1_QCDdatadriven2.png}
\includegraphics[width=0.49\textwidth]{QCDbginTauTau/Bin2_QCDdatadriven2.png} \\
\caption{Distributions of relevant kinematic variables before the requirement on the given variable
is applied: (a) \mttwo  (b) $\SumMT$ . The QCD multi-jet contribution is estimated from data using the ABCD method.}
\label{fig:5QCDbg}
\end{figure}

\subsection{\texorpdfstring{W+jets Background Estimation in $\tauTau$ Channel}{W+jets Background Estimation in tau-tau Channel}}
\subsubsection{Method Description}
As shown in the cut-flow-table, the number of W+jets events surviving the selection cuts are found to be zero in search \binone or 0.43$\pm$0.40 in search \bintwo. In other words, due to the low statistics of W+jets events, the efficiency of $\mttwo$ is obtained to be zero, in \binone, or as found in \bintwo, the efficiency of $\SumMT$ is obtained very close to zero with a huge statistical uncertainty, namely 93\%. The statistical uncertainty on the yields for W+jets events can be improved by extracting the $\mttwo$ or $\SumMT$ cut efficieny, depending on the serach bins, in a sample with more statistics. This region is defined by relaxing some cuts which have marginal effects on either $\mttwo$ or $\SumMT$ variables. This way, the yields for W+jets events are reported with more statistical accuracy. In the next subsection, the validation of this method will be discussed.\\

\subsubsection{Method Validation}
Figure~\ref{fig:justification_bin1}  
\begin{figure}[htbp]
\centering
\includegraphics[angle=0,scale=0.35]{TauTauFigs/withMT2GT40.png}
\includegraphics[angle=0,scale=0.35]{TauTauFigs/withMT2GT40ZVeto.png} \\
\caption{Cut efficiency for $\mttwo$ %(left) and \SumMT (right) 
in samples with various selections as labeled in each bin.}
\label{fig:justification_bin1}
\end{figure}
Figure~\ref{fig:justification_bin2}  
\begin{figure}[htbp]
\centering
\includegraphics[angle=0,scale=0.35]{TauTauFigs/WJetsEst_bin2.png}
\includegraphics[angle=0,scale=0.35]{TauTauFigs/WJetsEst_bin2_miscApplied.png} \\
\includegraphics[angle=0,scale=0.35]{TauTauFigs/WJetsEst_bin2_BJetVetoApplied.png} \\
\caption{Cut efficiency for %\mttwo (left) and 
$\SumMT$ %(right) 
in samples with various selections as labeled in each bin.}
\label{fig:justification_bin2}
\end{figure}
\subsection{\texorpdfstring{DY Background Estimation in $\tauTau$ Channel}{DY Background Estimation in tau-tau Channel}}
\subsubsection{Method Description}
\subsubsection{Method Validation}
\subsection{\texorpdfstring{Fake Background Estimation in $\leptonTau$ Channel}{Fake Background Estimation in lepton-tau Channel}}
\label{sect:bkgLeptau}
In the $e/\mu-\hadtau$ channels, the main background is the W +jets, when the $\W$ boson decays to a lepton and a jet fakes a hadronic $\tau$.
We use a fake rate method to estimate this background \cite{CMS_AN_2010-261}. 
The idea is that when the loose signal selection is applied, the number of the loose $\hadtau$'s (L) is:
\begin{equation}
L = P + F
\end{equation}
P is the number of the  prompt $\hadtau$'s and F is the number of the  fake $\hadtau$'s. If the selection is tightened, the number of the tight $\hadtau$'s (T) is:
\begin{equation}
 T = pP + fF
\end{equation} 
p (f) is the prompt (fake) rate, the probablity that a loosely selected prompt (fake) $\hadtau$ can pass the  tight  selection. The loose category (L) can be divided to two parts, 
tight (T) and non-tight (NT), so one can write:
\begin{equation}
   F * (f - p) = ((1 - p) * L - NT)
\end{equation}
f * F is the contamination of the fake $\hadtau$'s in the signal region. 

The fake rate ({\it f}) is measured as the ratio of the tightly selected $\hadtau$'s to the loosely 
selected $\hadtau$'s in a sample which is dominated by the fake $\hadtau$'s. The fake rate is estimated in an environment which is as similar as possible to 
the signal region. The datasets and the triggers which are used to estimate the fake rate in different channels are shown in 
table \ref{Tab.DataFR}.
\begin{table}[!htb]
\begin{center}
\caption{The datasets and triggers for fake rate estimation.}
\label{Tab.DataFR}
\begin{tabular}{|l|c|c|}
\hline
Channel      & Data Set                                     & Trigger \\\hline
Muon Tau     & /SingleMu/Run2012D-22Jan2013-v1/AOD          & HLT\_IsoMu24\_v(16-17)\\
             &                                              & HLT\_IsoMu24\_eta2p1\_v(14-15)\\\hline
Electron Tau & /SingleElectron/Run2012D-22Jan2013-v1/AOD    & HLT\_Ele27\_WP80\_v11\\
\hline
\end{tabular}
\end{center}
\end{table}

%In the muTau channel, exactly one muon is required which passes the selection criteria of the muon in the signal region and has \pt > 27 GeV. 
Lepton selection and extra lepton rejections are exactly same as the signal selection. Only the \pt of the favorite lepton is forced to 
be 3 \GeV higher than the online cut resulting to \pt $>$ 27 \GeV for muon and \pt $>$ 27 \GeV for electron.
The rejections can reduce the contribution of the VV, DY and $\ttbar$ events. To further suppress the $\ttbar$ contamination, the b-veto 
similar to the signal selection is applied. The \MET is asked to be greater than 30 GeV, similar to the preselections. The selection of the $\hadtau$ is 
exactly same as the signal selection, except the $\hadtau$ isolation which is {\it Loose} for the loosely selected $\hadtau$'s and {\it Tight} for the 
tightly selected $\hadtau$'s.
The ratio of these two categories determines the fake rate. To avoid any bias from the trigger, the $\hadtau$'s closer than $\Delta$R = 0.2 to the 
lepton are rejected. 
The fake rate can be measured in the bins of $\hadtau$ \pt and $\eta$, but it is observed that the dependency is very small and can be ignored, 
so we use a single value for the fake rate which is XXXX +- XXX.

The prompt rate (p) is measured in the MC DY events. All of the preselections except the Z-veto and $\hadtau$ isolation are applied. The $\hadtau$ isolation 
is relaxed from {\it Tight} to {\it Loose}. Only the events that a generated $\tau$ decays hadronically are considered. If a reconstructed $\hadtau$ is 
closer than $\Delta R = 0.1$ to the generated $\tau$, it is selected. Among these $\hadtau$'s, the prompt rate is defined as the fraction of the loose $\hadtau$'s 
which are tight. The prompt rate can be measured in the bins of \mttwo, but the statistics in the high \mttwo region which is our favorite 
region is too low that we can not conclude anything about the shape of the prompt rate in this region, so we choose it as a constant value
XXXX +- XX in the whole \mttwo region.




\section{Systematic uncertainties}
\label{sect:sys}
Systematic uncertainties can affect the shape or normalization of the
backgrounds estimated from Monte Carlo (\ttbar, Z+jets, dibosons and Higgs decays and \wjets in \tauTau \bintwo), 
as well as the signal acceptance. 

\subsection{b-jet veto uncertainty due to mismodelling of ISR}
The jet activity in our signal is just due to Initial State Radiation(ISR). The ISR is not simulated properly in our signal, because it has been generated using \PYTHIA $6.4$ event generator and it just produces $2 \rightarrow 2$ scattering processes. Therefore the ISR jet momenta spectra and its multiplicity cannot be trusted. It is possible that an ISR jet is mistagged as a b-jet and results in vetoing the signal event after applying the b-veto cut. On the other side \MADGRAPH ~\cite{MADGRAPH} generator, which is a matrix element generator provides a better description of ISR jets by supporting $2 \rightarrow 4$ interactions. To measure the amount of uncertainty introduced by b-veto cut we can compare the efficiency of this cut in our generated signal and its value in a sample similar to our signal which has been generated by \MADGRAPH. 
The most similar MC sample to our signal from the production point of view is WW sample. The $m(\chione) = 100\,\GeV$ and $m(\PSGczDo) = 1\,\GeV$ point of the signal is chosen so that masses resembles the WW production mechanism. % It is followed the below recipe, originating from $fixME$.
%By applying and relaxing the b-veto cut it is obtained the yields of WW and signal and by the uncertainty is A:
We obtain the efficiency of b-veto for signal sample denoted as $E^{SUSY}_{b-veto}$ and also for WW sample denoted as $E^{WW}_{b-veto}$.
Comparing these 2 efficiencies this uncertainty can be estimated.

\begin{align}
E^{WW}_{b-veto} &= \frac{WW_{0b}}{WW_{all}} = \frac{88968.51}{93058.33} = 0.95\\ \nonumber
\end{align}
and the b-veto efficiency for the signal is:
\begin{align}
E^{SUSY}_{b-veto} &= \frac{SUSY_{0b}}{SUSY_{all}} = \frac{2041.88}{2206.07} = 0.92 \\ \nonumber
\end{align}
and we take:
\begin{align}
|\frac{E^{SUSY}_{b-veto}-E^{WW}_{b-veto}}{E^{WW}_{b-veto}}| &= 3 \% \\ \nonumber
\end{align}

\subsection{\texorpdfstring{Uncertainty due to $\mindphifour$ cut}{Uncertainty due to minDeltaPhiJMET cut}}
Another source of systematic uncertainty due to existence of ISR in signal events is the cut on $\mindphifour$ ($ > 1$) against QCD events. We use the same method as mentioned above to obtain the systematic uncertainty of applying this variable. But this time we use Higgs to $\tau \tau$ sample which mimics our signal most.

\begin{align}
E^{Higgs}_{\mindphifour} &= \frac{Higgs_{\mindphifour>1.0}}{Higgs_{all}} = \frac{89288.14}{180562.56} = 0.49 \\ \nonumber
\end{align}
and this efficiency for the signal is: 
\begin{align}
E^{SUSY}_{\mindphifour} &= \frac{SUSY_{\mindphifour>1.0}}{SUSY_{all}} = \frac{1020.33}{2206.07} = 0.46 \\ \nonumber
\end{align}
and we take:
\begin{align}
|\frac{E^{SUSY}_{\mindphifour}-E^{Higgs}_{\mindphifour}}{E^{Higgs}_{\mindphifour}}| &= 6 \% \\ \nonumber
\end{align}

\subsection{\texorpdfstring{$\ell$ and \Tau energy scale}{Energy scale}}
The systematic uncertainty due to muon energy scale is small enough to be ignored.

The electron energy is varied by $1\%$ ($2.5\%$) for electrons reconstructed in the barrel (endcap) region as recommended by EGAMMA POG\cite{eEnergyScale}. As this variation has small effects on electron $\pt$ related variables, its effect on final yields in \eTau channel has been found negligible.

The energy of \Tau's is scaled by $3\%$, following the recommendation of the Tau POG~\cite{TauPOG}. Variables like \MPT, \mttwo, \mindphifour and invariant mass which depend on \Tau $\pt$ are re-calculated.  Up and down uncertainties in the final yields are reported in table~\ref{Tab.tauEnergyScale}. 
\begin{center}
\begin{table}[!Hhtb]
\scriptsize{
\caption{Tau energy scale systematic effect on MC in different channels. The statistical uncertainty is also quoted to be able to guess the statistical contamination in the systematic values.}
\begin{tabular}{|c|c|c|c|c|c|c|}
\hline  
                            & ZX    & Higgs  & WW   & Top    & All MC & SUSY (380 , 1)%& Data%
 \\\hline 
$e\Tau$ channel            & $0.38\pm0.06^{+0.13}_{-0.08}$ & $0.06\pm0.02^{+0.0}_{-0.02}$  & $0.05\pm0.04^{+0.0}_{-0.0} $ &$0.02\pm0.02^{+0.02} _{-0.0}$  & $0.45\pm0.07^{+0.14}_{-0.03}$ & $2.14 \pm 0.10 ^{+0.07} _{-0.05} $ %& $3.0\pm1.73 ^{+0.0 \%} _{-33 \%}$
    \\\hline   
%^{+0.14}_{-0.03}
%W:$1.29\pm0.62^{+0.0}_{-0.0} $
$\mu\Tau$ channel      &  $0.28 \pm 0.05 ^{+0.14} _{-0.06} $      & $0.05\pm0.02^{+0.03}_{-0.0}$   & $0.34 \pm 0.14 ^{+0.37} _{-0.24} $        &  $0.0\pm0.0 ^{+0.67} _{-0.06} $   &    $0.66  \pm 0.15 ^{+0.34} _{-0.13} $      &  $2.16 \pm 0.11^{+0.09} _{-0.06} $      %& $5.0 ^{} _{} $     
\\\hline  
%$0.79 \pm 0.47^{+0.8} _{-0.35} $
$\tauTau$ \binone     &    $0.56 \pm 0.07 ^{+0.7} _{-0.09}$    & $0.17 \pm 0.04 ^{+0.05} _{-0.05}$       &  $0.02 \pm 0.02 ^{+0.0} _{-0.02}$        &   $0.0 \pm 0.0 ^{+0.0 } _{-0.0 }$        &    $0.75 \pm 0.08 ^{+0.21} _{-0.19}$     & $4.10 \pm 0.38^{+0.05} _{-0.03} $    %& $1.0 \pm1.0 ^{+0.0 \%} _{-0.0 \%}$
\\\hline
%$0.0 \pm 0.0 ^{+0.0} _{-0.0}$
$\tauTau$ \bintwo    &     $0.81 \pm 0.56 ^{+0.39} _{-0.7}$     &   $0.07 \pm0.02 ^{+0.02} _{-0.03}$      &     $0.15 \pm 0.07 ^{+0.0} _{-0.02}$     &   $0.53 \pm 0.53 ^{+0.0} _{-0.0}$   &      $1.48 \pm 0.77 ^ {+0.49} _{-0.28}$     &     $1.10 \pm 0.07 ^{+0.04} _{-0.02}$   %&  $2.0 \pm 1.41 ^{} _{}$   
 \\\hline
%$0.43 \pm 0.4 ^{+0.0} _{-0.0 }$
\end{tabular} 
\label{Tab.tauEnergyScale}
}
\end{table}     
\end{center}
The big value of the \Tau energy scale uncertainty is dominated by the lack of the statistics. In signal samples which have enough statistics the uncertainty decreases. 
In some cases which the statistical uncertainty had a large value comparing to systematic uncertainty, 
we relaxed some \Tau \pt  independent variables to obtain more statistics. 
In another approach, the \pt related cuts are applied one by one and uncertainty of each cut is calculated indepedently. 
Finally, all individual uncertainties are taken to be independent and added in quadrature. The final uncertainties due to
tau energy scale vary in the range of 10-15\% and 2-15\% in different signal regions for backgrounds and signals, respectively.
%Eskandari's presentation 8 April 2015

%So the maximum uncertainty due to \Tau energy scale in MC driven backgrounds and signal is set to $~10\%$.
%In order to calculate systematic uncertainties for signal, we used 3 signal points which are ($m(\chione) = 180\,\GeV$, $m(\PSGczDo) = 60\,\GeV$), (240, 40) and (380, 1) representing low, moderate and high delta mass respectively. 
%The $10\%$ effect of this uncertainty for all signal points in different channels is a conservative value that covers the whole plain and all channels.


\subsection{Monte Carlo  statistics} 
The statistics in the simulated Monte Carlo samples are also a
  source of the  uncertainties. % which are 20\% for the background processes and 10\% for the signal events.
This uncertainty varies for signal points between 3\% and 15\% and for the backgrounds between 13\% and 70\%.

\subsection{b-jet ID}
The uncertainty due to the b-tag scale factor on the signal and background events is taken into account by varying the scale factors within their 
uncertainty. It is found that for signal, there is an uncertainty of about $8\%$. For almost all of the backgrounds except for backgrounds including Top, an uncertainty around 1\% is found. For Top backgrounds, the uncertainty is about 4\%. Therefor, for background events at most a 4\% uncertainty due to the b-veto cut is assigned. 
 
\subsection{Lepton trigger, identification and isolation efficiency}
The uncertainties in electron and muon triggers, identification and isolation efficiencies are $2\%$ for electrons and muons ~\cite{CMS_AN_2013-171}. The uncertainty in the \Tau identification efficiency amounts to $6\%$ ~\cite{CMS_AN_2013-171}
The uncertainty in the efficiency of the hadronic tau leg of the $e\Tau$ and $\mu\Tau$ ($\tauTau$) trigger amounts to $3.0\%$ ($4.5\%$ per leg).

\subsection{PDF}
The signal acceptance changes due to PDF uncertainties is expected to be small. We take this uncertainty from ~\cite{CMS_AN_2012-248}, where the signal has been produced through electroweak process like our signal. The amount of this uncertainty is around $2\%$.

%The effect of PDF on cross section is considered using $CTEQ66$ and $MSTW2008nlo90cl$ for PDF and it is shown for some generic SUSY points in Table.~\ref{Tab.PDF}.
%\begin{table}[!Hhtb]
%\begin{center}
%\caption{Cross section systematic uncertainty due to different PDF.}
%\begin{tabular}{|c|c|c|}
%\hline
%                                    &$\sigma (fb) \_ CTEQ66$          & $\frac{\sigma \_ CTEQ66 - \sigma \_ MSTW2008}{\sigma \_ CTEQ66}$  \\\hline 
%m(\chione) = 100 GeV                &$5823.40^{+0.0 \% + 3.4 \%}_{-0.6 \% - 3.2 \%}$         & 3 \%         \\\hline   
%m(\chione) = 200 GeV                &$379.24^{+0.4 \% + 4.5 \%}_{-0.4 \% - 4.4 \%}$          & 6 \%        \\\hline  
%m(\chione) = 300 GeV                &$67.51^{+0.2 \% + 5.9 \%}_{-0.2 \% - 5.1 \%}$           & 7 \%        \\\hline
%m(\chione) = 400 GeV                &$17.51.40^{+0.0 \% + 6.8 \%}_{-0.3 \% - 6.3 \%}$        & 8 \%        \\\hline
%m(\chione) = 500 GeV                &$5.53^{+0.0 \% + 8.1 \%}_{-0.9 \% - 7.0 \%}$            & 12 \%        \\\hline
%\end{tabular} 
%\label{Tab.PDF}
%\end{center}
%\end{table}     

\subsection{Luminosity}                                                                                                                                                           
The uncertainty in the luminosity  is $2.6\%$ for $2012$ data ~\cite{LUMI}.                                                                                                       
This affects mainly the  normalization of the signal Monte Carlo samples, because for the backgrounds  either  the data-driven methods are used or
the normalization is found from data.

\subsection{Pile-up}
The minimum bias cross section is varied $5 \%$ up and down following the standard recipe of ~\cite{PU_SYS}. It is found to introduce $~4 \%$ systematic for all channels.    

\subsection{\texorpdfstring{\MPT}{MET}}
The main backgrounds come from data driven or MC validated against data. \MPT uncertainties can be important for signal only. Our signal is produced through an SUSY EWK production and does not include jets directly but it is expected to feature four invisible particles as genuine missing energy. Therefore it should not be affected by jet calibrations. Apart from jets, lepton and tau energy scales and their effects on \MPT have been already taken into account. There is just unclustered energy left which based on results in ~\cite{CMS_AN_2014-099} this item's effect on \MPT is small enough to be ignored.



\subsection{\texorpdfstring{Inefficiency due to a bug in \Tau trigger}{Inefficiency due to a bug in tau trigger}}
To consider the effect of a known trigger bug in $\tauTau$ channel~\cite{CMS_AN_2014-074}, the $\pt$ 
distribution of the leading \Tau in \binone for the signal SMS point corresponding to (380,1) is shown (in black) in figure~\ref{fig:tauPt}.
\begin{figure}[!Hhtb]
\centering
\includegraphics[angle=0,scale=0.35]{TauTauFigs/leadingTauPt.png}
\caption{The $\pt$ distribution of the leading \Tau in \binone of the $\tauTau$ channel. Also shown is the modified $\pt$ distribution after correction for the trigger bug.}
\label{fig:tauPt}
\end{figure}

 The choice of this signal point is because of the fact that this SMS point can provide the maximum efficiency of 
having \Tau with $\pt$s above 140 \GeV.  In the same plot, the modified pt-distribution is shown in red, 
which is obtained as follows. Based on figure~35 of the same reference, a correction factor is taken and 
multiplied bin-by-bin to the black histogram (for the bins above 140 GeV). The difference in integral
of the two histograms is of the order of ~1.5\% (original integral equals 4.10, while the modified integral is equal to 4.04).
Due to smallness of this effect comparing to other systematic, it is ignored and no correction is applied. 


\subsection{\texorpdfstring{\Tau efficiency in fast simulation}{Tau efficiency in fast simulation}}
The fast simulation shows some differences with the full simulation, especially in track reconstruction. It can affect the \Tau isolation.
To evaluate the effect of this inefficiency, the \Tau isolation/identification efficiency  is compared in the fast and full simulation.
Figure \ref{fig:TauEffFastFull} (left)
\begin{figure}[!Hhtb]
\centering
\includegraphics[angle=0,scale=0.35]{SystematicFigs/TauEff_lepTau.pdf}
\includegraphics[angle=0,scale=0.35]{SystematicFigs/TauEff_lepTau_ratio.pdf}
\caption{Comparison of the \Tau efficiency in the full and fast simulation (left). The ratio of the efficiencies in the fast and full simulation. (right)}
\label{fig:TauEffFastFull}
\end{figure}
compares the \Tau efficiency in WWjets and DYjets which are full simulation with signal which is fast simulation. In 
both cases, only \Tau in \leptonTau channels are considered. In 
\ref{fig:TauEffFastFull} (right), the ratio of the efficiency of the fast simulation to full simulation is shown. 
The main source of the difference, is the difference between the additional event activities in the samples, so we assign 5\% systematic uncertainty per \Tau leg.

\subsection{Low rate backgrounds} For some backgrounds like \ttbar, dibosons and Higgs decays, the remaining 
events from the simulation are very small. A 50\% uncertainty is considered for these backgrounds to account for the theoretical uncertainty of the
cross section calculation as well as the shape mismodeling.

\subsection{Summary}
The summary of all systematic uncertainties for different channels is reported in Table.~\ref{Tab.SYS}. The main source of the systematic uncertainty in all channels and samples is the \Tau energy scale.  
The syatematic uncertainties that can alter the shapes are added in quadrature and 
treated correlated when two signal regions of \tauTau channel are combined. Other systematic unceratinties of these two 
cahnnels and all of the systematic unceratinties of \leptonTau channels are treated uncorrelated.

%We add all systematic uncertainties in quadrature and assign 
% 20\% and 25\% relative uncertainties in the signal
%acceptance for the \leptonTau and \tauTau channels, respectively. For the Monte Carlo background predictions, the values are 
%25\% and 28\%, respectively. The relative uncertainty of the less important backgrounds is 50\% for all channels.

\begin{table}[!htb]
\begin{center}
\caption{Summary of systematic uncertainties that affect the signal event selection efficiency and the background normalization and their shape. The sources that alter
the shape are indicated by (*) next to their names. The shape-altering sources are considered correlated between two signal regions of \tauTau in the final statistical combination.}
\small{
\begin{tabular}{|l|ccc|ccc|}
\hline\hline
                              &\multicolumn{3}{c|}{Background}         &\multicolumn{3}{c|}{Signal}\\\hline
                              &            & \tauTau & \tauTau         &            & \tauTau & \tauTau\\
Systematic uncertainty source & \leptonTau & \binone &  \bintwo        & \leptonTau & \binone &  \bintwo        \\
\hline\hline
%*\Tau energy scale&\multicolumn{3}{c|}{10} &\multicolumn{3}{c|}{10} \\\hline
\Tau energy scale (*)&10\% &\multicolumn{2}{c|}{15\%}  & 2-12\% &\multicolumn{2}{c|}{3-15\%} \\\hline 
\Tau id efficiency& 6\% &\multicolumn{2}{c|}{12\%} & 6\% &\multicolumn{2}{c|}{12\%}  \\\hline
\Tau trigger efficiency& 3\%&\multicolumn{2}{c|}{9\%}& 3\%&\multicolumn{2}{c|}{9\%}  \\\hline
Lepton trigger, id, iso efficiency& 2\% & \multicolumn{2}{c|}{-} & 2\% &  \multicolumn{2}{c|}{-} \\\hline
\MPT (*)&\multicolumn{3}{c|}{5\%} &\multicolumn{3}{c|}{5\%} \\\hline
b-tagged jets veto & 4\% & - & 4\% &  8\% & - & 8\% \\\hline
Pile-up&\multicolumn{3}{c|}{4\%} &\multicolumn{3}{c|}{4\%} \\\hline
Fast/Full \Tau id efficiency &\multicolumn{3}{c|}{-}& 5\% & \multicolumn{2}{c|}{10\%}\\\hline
ISR (*)&\multicolumn{3}{c|}{-}&\multicolumn{3}{c|}{3\%} \\\hline
\mindphifour&\multicolumn{3}{c|}{-}&\multicolumn{3}{c|}{6\%} \\\hline
PDF (*)&\multicolumn{3}{c|}{-}&\multicolumn{3}{c|}{2\%} \\\hline
Luminosity       &\multicolumn{3}{c|}{-} & \multicolumn{3}{c|}{2.6\%}\\\hline
Total shape-altering sys. & 11\% & 16\% & 16\% & 6-13\% &\multicolumn{2}{c|}{7-16\%} \\\hline
Total non-shape-altering sys. & 9\% & 16\% & 16\% & 14\% &20\%& 21\% \\\hline
Total Systematic&  14\% & 22\%  & 22\%& 15-19\% & 21-25\%  & 22-26\%\\\hline
Monte Carlo Statistic & 22\% & 13\% & 70\% & \multicolumn{3}{c|}{3-15\%} \\\hline
Total& 26\% & 26\%  & 73\%& 15-24\% & 21-29\%  & 22-30\%\\\hline
Low rate backgrounds &\multicolumn{3}{c|}{50\%}&\multicolumn{3}{c|}{-}\\\hline
\hline
\end{tabular}
}
\label{Tab.SYS}
\end{center}
\end{table}


\section{Summary and statistical interpretation of the results}
\label{sect:stat}
To interpret the results, we first recapitulate a little of assumptions and the consequent numbers.
In this analysis, we examine the data in three different channels.
These channels include \tauTau, \muTau and \eTau.
In the \muTau and \eTau channels, one signal region is defined for each channel , which is $\mttwo > 90$ \GeV and $\tauMT > 200$ \GeV,
%Due to the sensitivity of \tauTau channel to the signal, we look at the data in two different bins.
but the \tauTau channel has two signal regions.
The \binone is $\mttwo > 90$ \GeV and the \bintwo is $40 < \mttwo < 90$ \GeV and $\SumMT > 250$ \GeV.
We eventually combine all four bins to utilize more information from the observed and the predicted distributions.
It was checked that applying the cuts of each channel on other channels do not remain any event and 
there is not any overlap between the channels, so the channels can be statistically combined.

Figure \ref{fig:yield_final}
\begin{figure}[h]
\centering
%\includegraphics[width=0.35\textwidth,keepaspectratio=true]{StatisticsFig/QCDWestimation_plot.png}
%\includegraphics[width=0.35\textwidth,keepaspectratio=true]{StatisticsFig/QCDestimation_plot.png}
\includegraphics[width=0.35\textwidth,keepaspectratio=true]{StatisticsFig/MT2_tauMTgt200_DDFake.png}
\includegraphics[width=0.35\textwidth,keepaspectratio=true]{StatisticsFig/MT2_tauMTgt200_DDFake.png}
\caption{The comparison of data and MC expectation in \leptonTau channels.}
\label{fig:yield_final}
\end{figure}
compares the data and the MC expectation in different \leptonTau channels. 
In these plots, the QCD multijet and \wjets and fake contribution from other channels is shown 
as ``Fake'' which was described in Sec.\ref{sect:bkgFake}. In Fig.\ref{fig:5QCDbg}, the \mttwo and \SumMT variables in two 
different signal regions of \tauTau channel are shown. The QCD multijet contribution in these plots come from the data driven method described in 
Sec.\ref{sect:bkgQCD}. The \wjets in the last bin of the top-left plot was described in Sec.\ref{sect:bkgW}. 

Backgrounds are taken into account in different categories, including Monte-Carlo-Driven, Low rate backgrounds, Fake and QCD which are data driven.
%However, due to the method of background estimation in the \tauTau channel \binone,  this channel has one more category 
%called W-jets (W).
As a summary of results, the data and background yields and systematic uncertainties are listed in Table \ref{tbl:yieldSysSummary}. 
\begin{table}[!htb]
\begin{center}
\begin{small}
\caption{Data yields and background predictions with uncertainties in the four signal regions of the search. 
The uncertainties are reported in two parts, the statistical and systematic uncertainties, respectively. 
The \wjets and QCD multijet main backgrounds are derived from data as described in Section~\ref{sect:bkg}; 
the abbreviation ``VV'' refers to diboson events. The yields for three signal points representing the low, medium, and high $\Delta m$
are also shown. SUSY(X, Y) stands for a SUSY signal with ${\rm{m}}_{\chione}$ = X\GeV and ${\rm{m}}_{\neutralino}$ = Y\GeV.}
\begin{tabular}{|c|c|c|c|c|}
\hline
	           & \eTau & \muTau & \tauTau \binone & \tauTau \bintwo \\
\hline
  DY               & 0.19 $\pm$ 0.04 $\pm$ 0.03 & 0.25 $\pm$ 0.06  $\pm$ 0.04  &  0.56 $\pm$ 0.07 $\pm$ 0.12 & 0.81 $\pm$ 0.56 $\pm$ 0.18  \\
tX, VV, hX  & 0.03 $\pm$ 0.03 $\pm$ 0.02 & 0.19 $\pm$ 0.09  $\pm$ 0.09  &  0.19 $\pm$ 0.03 $\pm$ 0.09 & 0.75 $\pm$ 0.35 $\pm$ 0.38  \\
\wjets             & 3.30$_{- 3.30}^{+ 3.35}$ $\pm$ 0.56 & 8.15 $\pm$ 4.59  $\pm$ 1.53  &  0.70 $\pm$ 0.21 $\pm$ 0.55 & 4.36 $\pm$ 1.05 $\pm$ 1.63  \\
QCD multijet       &             -              &            -                 &  0.13 $\pm$ 0.06 $\pm$ 0.21 & 1.15 $\pm$ 0.39 $\pm$ 0.74  \\
\hline
SM total           & 3.52 $\pm$ 3.35 $\pm$ 0.56 & 8.59 $\pm$ 4.59  $\pm$ 1.53  &  1.58 $\pm$ 0.23 $\pm$ 0.61 & 7.07 $\pm$ 1.30 $\pm$ 1.84  \\
\hline
Observed           &               3            &                5             &             1               & 2     \\\hline  
SUSY(380, 1)        & 2.14 $\pm$ 0.08 $\pm$ 0.38 & 2.16 $\pm$ 0.08  $\pm$ 0.39  &  4.10 $\pm$ 0.10 $\pm$ 0.90 & 1.10 $\pm$ 0.05 $\pm$ 0.27 \\
SUSY(240, 40)       & 1.43 $\pm$ 0.19 $\pm$ 0.21 & 0.96 $\pm$ 0.14  $\pm$ 0.14  &  4.35 $\pm$ 0.27 $\pm$ 0.91 & 3.60 $\pm$ 0.25 $\pm$ 0.83 \\
SUSY(180, 60)       & 0.12 $\pm$ 0.04 $\pm$ 0.02 & 0.04 $\pm$ 0.02  $\pm$ 0.01  &  0.73 $\pm$ 0.11 $\pm$ 0.17 & 2.36 $\pm$ 0.17 $\pm$ 0.54 \\
\hline
\end{tabular}
\label{tbl:yieldSysSummary}
\end{small}
\end{center}
\end{table}

The last row of the table shows the observed data for  each individual channel.  The uncertainties are systematic, unless when there are 
two parts, the first part is statistics.
As seen in the table, the overall background yields of all categories, 
for the \eTau, \muTau, \tauTau \binone and \bintwo are  
2.96, 7.28, 1.57 and 3.14, respectively.
MC Driven backgrounds are feed to the package as a Gamma distribution with the corresponding statistical weights for each bin.
Furthermore, 25\% systematic uncertainty on MC Driven backgrounds is also considered. These systematic uncertainties are considered uncorrelated.
All systematics are taken through LogNormal distributions. For the Low rate backgrounds, a flat 50\% uncertainty is assigned.
%Systematic uncertainties for each bin of DD backgrounds are 170\%, 79\%, 50\% and 69\% respectively. 
The QCD background estimations for 
both bins of \tauTau channel use the same category of events, so the estimation in two bins are treated 100\% correlated. 
The systematic uncertainties for the fake estimation in $\ell\Tau$ channels are also 100\% correlated. 
20\% systematic uncertainty on signal yields is considered. 
Due to the large signal sample which is used, no statistical uncertainty is assigned to signal.

Since no excess of data over the background prediction is observed, 
we close our study with setting upper limits on the testing signals.
This is conducted using a modified frequentist approach, namely CLs method \cite{read:CLs}.
In this method, the test statistic $q_\mu$ 
%\cite{cowan:asymptoticCLs} 
is a function of the profile likelihood-ratio,

\begin{align}
q_\mu = -2 \ln \frac{\mathcal{L}(data ;\, b + \mu s)}{\mathcal{L}(data ;\, b + \hat{\mu} s)},
\end{align}

where $\hat\mu$ is the \textit{signal strength modifier} $\mu$ at the maximum point of the likelihood $\mathcal{L}$.
Then CLs is given by the following probability-ratio,

\begin{align}
CL_s = \frac{p(q_\mu \geq q_\mu^{obs} | b + \mu s )}{p(q_\mu \geq q_\mu^{obs} | b)}.
\end{align}
 
We compute CLs using a software package provided by the CMS Higgs PAG \cite{higgspag:software}.
After incorporating systematic uncertainties, an observed CLs smaller than 0.05 for a signal strength of $\mu = 1$, excludes the given signal at $95\%$ CL. Indeed, the package determines which signal strength $\mu$ excludes the testing signal at $95\%$ CL. Therefore all resulting $\mu \leq 1$ define the excluded region in the parameter space of the given signal. 

%To investigate the exclusion power of our research, we study the topology of direct stau pair production and the \PSGcpDo\PSGcmDo production in Simplified Models \cite{alves:sms}. 
%This research deal with tau family decay of charginoes including 
%$ \chione \rightarrow \sTau + \nu ~~\mathrm{and}~~  \chione \rightarrow \sNu_{\tau} + \tau $.
%As discussed in Section \ref{sect:introduction}, the final state is full of $\tau$ and $ \sTau \rightarrow \tau + \PSGczDo  $.
%Hence many channels could be define, due to decays of tau to electrons, muons and hadrons.    

Panels represented in figure \ref{fig:limit_bins}
%%%%%%%%%%
\begin{linenomath}
\begin{figure}[h]
\centering
\includegraphics[width=0.49\textwidth,keepaspectratio=true]{StatisticsFig/Exclusion_TauTauBin1.png}
\includegraphics[width=0.49\textwidth,keepaspectratio=true]{StatisticsFig/Exclusion_TauTauBin1_Bin2.png}
\includegraphics[width=0.49\textwidth,keepaspectratio=true]{StatisticsFig/Exclusion4Bins_MuTauExcl.png}
\includegraphics[width=0.49\textwidth,keepaspectratio=true]{StatisticsFig/Exclusion4Bins_EleTauExcl.png}
\caption{These figures show the impact of each bin on the final combination. 
The top panels are related to the \tauTau channel including the \binone alone (left) and combination of \binone and \bintwo (right).
The bottom ones show the expected and observed exclusion limit when \eTau (left) and \muTau (right) channels are included in the \tauTau channel.
}
\label{fig:limit_bins}
\end{figure}
\end{linenomath}
%%%%%%%%%%
 show the impact of each bin on the combined result, represented by the final exclusion limit shown in figure \ref{fig:limit_final}. 
%%%%%%%%%%
\begin{linenomath}
\begin{figure}[h]
\centering
\includegraphics[width=0.7\textwidth,keepaspectratio=true]{StatisticsFig/Exclusion4Bins.pdf}
\caption{Expected exclusion power in terms of Simplified Models
with the total dataset of 2012. The observed plot is also shown.
%Backgrounds are predicted using Monte-Carlo simulations and a rough estimate of systematic uncertainties equal $10\%$ is taken into account.
}
\label{fig:limit_final}
\end{figure}
\end{linenomath}
%%%%%%%%%%
The top-left panel of figure \ref{fig:limit_bins} shows the expected and observed exclusion region in the plane of $m_{\chione}-m_{\PSGczDo}$
calculated by the simulated samples in the first bin of \tauTau channel. The top-right panel in figure \ref{fig:limit_bins} 
is produced by using both bins of \tauTau channel.
As seen, the inclusion of \bintwo of the \tauTau channel causes a little expansion of the exclusion limit towards 
the diagonal (low mass difference).
Two bottom panels of figure \ref{fig:limit_bins} show the exclusion limits when \eTau (left) and \muTau (right) channels are 
added to the \tauTau channel.  As seen, these channels individually improve the limit on the right side of plane (high mass difference).
In order to compute quickly limits, the asymptotic CLs method is used to prepare the plots of figure \ref{fig:limit_bins}.
The hybrid method is reasonably more accurate than the asymptotic method. Hence, the final exclusion limit is computed by both of the methods.
On comparison, their results were fairly the same. 
Nevertheless, the final results in figure \ref{fig:limit_final} are computed by the hybrid method.
Figure \ref{fig:limit_final} shows the expected and observed exclusion limits 
of the chargino pair production in terms of Simplified Models \cite{alves:sms}. 
Calculation of the expected exclusion limits shows that the search has a potential to exclude 
a sizable region of the phase space, surrounded by the lines of $m_{\chione} = 410\GeV$ and $m_{\PSGczDo} = 100\GeV$ with 
the total dataset of 2012. The boundaries are well beyond the ATLAS reach which is \chione  masses up to 345 \GeV \cite{Aad:2014yka}.
Adding \leptonTau channels
and considering two signal regions in \tauTau channel has increased the sensitivity of the current search compared to ATLAS which uses only 
one search region in \tauTau channel to make the exclusion.
The observed limit excludes the $\chione$ with masses up to 420 \GeV when the $\PSGczDo$ mass is zero.
The \sTau searches in the LEP experiments \cite{lepsusy} have excluded the masses below 95 \GeV. In Fig.~\ref{fig:limit_final}, 
this region corresponds to the triangle in bottom-left corner.
The diagonal line denotes the boundary for $m_{\chione} = m_{\tau} + m_{LSP}$, which is the kinematical boundary of the search.
The expected limits and their one standard deviations introduced by the experimental 
uncertainties are shown with the red solid and dashed lines, respectively. The observed limits are shown with the black solid lines, the one 
standard deviations are shown with narrower black lines. The theoretical cross sections are moved up and down by one one standard deviation to 
find the narrow lines.
The signal cross sections in NLO + next-to-the-leading-logarithm (NLL) order in $\alpha_s$ are used to make the exclusion limits.
In the whole region, the observed limits lie closer than one standard deviation from the expected limits.  



The results of the \tauTau channels are interpreted to set limit on the $\tilde{\tau}\tilde{\tau}$ production, which corresponds to the right diagram in Fig.~\ref{fig:Productions}. In this simplified model, two $\tilde{\tau}$ are directly produced from the $pp$ collision and decay instantly into two $\tau$ and two \PSGczDo. As the cross section of direct production of sleptons is lower,
\begin{linenomath}
\begin{figure}[h]
\centering
\includegraphics[width=0.5\textwidth,keepaspectratio=true]{StatisticsFig/ExclusionSTauSTauLsp1.pdf}
\caption{The exclusion power of the \tauTau channels in $\tilde{\tau}\tilde{\tau}$ production.}
\label{fig:limit_stau_stau}
\end{figure}
\end{linenomath}
 no point is excluded and $95\%$ upper limit is set on the cross section. Adding two $\ell\Tau$ channels does not improve the results.
Figure ~\ref{fig:limit_stau_stau} represents the ratio of the 
obtained upper limit on the cross section and the cross section expected from SUSY (signal strength) vs. the mass of the $\tilde{\tau}$ particle, when \PSGczDo mass is $1 GeV$.
The observed ratio is within one standard deviation of  the expected ratio.
The best limit which corresponds to the lowest signal strength is obtained for $m_{\tilde{\tau}}=110 GeV$. The observed (expected) upper limit on the cross section at this mass is 232 (289) $fb$ which is almost 3 times of the theoretical NLO cross section.

\section{Information to test the new models}
\label{sect:model}
In the previous sections, a simplified SUSY model was used to optimize the selections and interpret the results. 
Here, the main efficiencies are reported versus the generated values, that can be used to examine the new models approximately in 
a MC generator-level study. The number of the remaining signal events and its uncertainty which can be evaluated by a generator-level study 
should be combined statistically with the results in Tab. \ref{tbl:yieldSysSummary} to find the upper limit on the number of the signal events
and decide if a model is excluded or still allowed in the shadow of the analysis presented in the paper.

In different channels, the generated taus are found. If it decays to leptons, 4-vector of the leptons and if it decays to hadrons, the difference between the 4-vector 
of the generated tau and the $\nu_{tau}$ are used as the generated particles to parametrize corresponding reconstructed particle. The latter 4-vector is referred to as the
visible \Tau. Table \ref{tbl:EffTauLep}
\begin{table}[!Hhtb]
\begin{center}
\begin{tabular}{|c|c|c|c|c|c|}
\hline\hline
generated \pt (\GeV)  & e for $e\Tau$ & $\mu$ for $\mu\Tau$  & \Tau for $\ell\Tau$    &  $\Tau^1$ for \tauTau & $\Tau^2$ for \tauTau\\
\hline\hline
0-10            &    0.1        &   0.1                &  0.1                   &       0.1           & 0.1\\\hline
10-20           &    0.1        &   0.1                &  0.1                   &       0.1           & 0.1\\\hline
20-30           &    0.1        &   0.1                &  0.1                   &       0.1           & 0.1\\\hline
30-40           &    0.1        &   0.1                &  0.1                   &       0.1           & 0.1\\\hline
40-60           &    0.1        &   0.1                &  0.1                   &       0.1           & 0.1\\\hline
60-80           &    0.1        &   0.1                &  0.1                   &       0.1           & 0.1\\\hline
80-120          &    0.1        &   0.1                &  0.1                   &       0.1           & 0.1\\\hline
120-160         &    0.1        &   0.1                &  0.1                   &       0.1           & 0.1\\\hline
160-200         &    0.1        &   0.1                &  0.1                   &       0.1           & 0.1\\\hline
$>$ 200         &    0.1        &   0.1                &  0.1                   &       0.1           & 0.1\\\hline
\hline
\end{tabular}
\caption{Efficiency to select a lepton or \Tau in different channels. $\Tau^1$ and $\Tau^2$ stand for leading and next-to-leading \Tau in the \tauTau channel.}
\label{tbl:EffTauLep}
\end{center}
\end{table}
shows the efficiency of selecting a lepton or \Tau for different channels versus the \pt of the generated lepton or visible \Tau. When \tauTau  or $\ell\Tau$ are selected, 
the negative of the sum of the 4-vector of the pair is used as the 4-vector of the generated missing particles. Its \pt is used as the generated \MET. Table \ref{tbl:EffMet}
\begin{table}[!Hhtb]
\begin{center}
\begin{tabular}{|c|c|c|}
\hline\hline
generated \MET (\GeV)  & $\ell\Tau$  &  \tauTau \\
\hline\hline
0-10            &    0.1        &   0.1   \\\hline
10-20           &    0.1        &   0.1   \\\hline
20-30            &    0.1        &   0.1   \\\hline
30-40            &    0.1        &   0.1   \\\hline
40-50            &    0.1        &   0.1   \\\hline
50-60            &    0.1        &   0.1   \\\hline
60-70            &    0.1        &   0.1   \\\hline
70-80            &    0.1        &   0.1   \\\hline
80-90            &    0.1        &   0.1   \\\hline
90-100            &    0.1        &   0.1   \\\hline
100-120            &    0.1        &   0.1   \\\hline
120-140            &    0.1        &   0.1   \\\hline
140-160            &    0.1        &   0.1   \\\hline
160-200            &    0.1        &   0.1   \\\hline
$>$ 200            &    0.1        &   0.1   \\\hline
\hline
\end{tabular}
\caption{Efficiency to pass the cut on \MET ($>$ 30 \GeV) in different channels versus the generated \MET.}
\label{tbl:EffMet}
\end{center}
\end{table}
shows the efficiency in different channels to pass the \MET $>$ 30 \GeV as a function of the generated \MET. The mass of the system of the selected pair is used to parametrize 
the efficiency to pass the cuts on the reconstructed invariant mass. Table \ref{tbl:EffMass}
\begin{table}[!Hhtb]
\begin{center}
\begin{tabular}{|c|c|c|}
\hline\hline
generated mass (\GeV)  & $\ell\Tau$  &  \tauTau \\
\hline\hline
0-5            &    0.1        &   0.1   \\\hline
5-10         &    0.1        &   0.1   \\\hline
10-15         &    0.1        &   0.1   \\\hline
15-20         &    0.1        &   0.1   \\\hline
20-25         &    0.1        &   0.1   \\\hline
25-30         &    0.1        &   0.1   \\\hline
30-35         &    0.1        &   0.1   \\\hline
35-40          &    0.1        &   0.1   \\\hline
40-45          &    0.1        &   0.1   \\\hline
45-50          &    0.1        &   0.1   \\\hline
50-55          &    0.1        &   0.1   \\\hline
55-60          &    0.1        &   0.1   \\\hline
60-65          &    0.1        &   0.1   \\\hline
65-70          &    0.1        &   0.1   \\\hline
70-75           &    0.1        &   0.1   \\\hline
75-80          &    0.1        &   0.1   \\\hline
80-85         &    0.1        &   0.1   \\\hline
85-90         &    0.1        &   0.1   \\\hline
90-95         &    0.1        &   0.1   \\\hline
95-100         &    0.1        &   0.1   \\\hline
$>$ 100         &    0.1        &   0.1   \\\hline
\hline
\end{tabular}
\caption{Efficiency to pass the cut on the invariant mass in different channels versus the generated mass.}
\label{tbl:EffMass}
\end{center}
\end{table}
shows the efficiency in different channels to pass the cut on the invariant mass ($>$ 15 \GeV) and ( $<$ 45 or $>$ 75 \GeV) for the $\ell\Tau$ channels 
and ( $<$ 55 or $>$ 85 \GeV) for the \tauTau channel. To parametrize the efficiency to pass the selection cuts on \mttwo, 4-vector of the generated lepton, \Tau and 
missing particles is used to calculate the generated \mttwo. The efficiency to pass the cut (\mttwo $>$ 90 \GeV) in $\ell\Tau$ and \tauTau \binone is shown in Tab. \ref{tbl:EffMT2}. 
\begin{table}[!Hhtb]
\begin{center}
\begin{tabular}{|c|c|c|}
\hline\hline
generated \mttwo (\GeV)  & $\ell\Tau$  &  \tauTau SR1 \\
\hline\hline
0-20    &    0.1        &   0.1   \\\hline
20-40 &    0.1        &   0.1   \\\hline
40-50 &    0.1        &   0.1   \\\hline
50-60 &    0.1        &   0.1   \\\hline
60-70 &    0.1        &   0.1   \\\hline
70-80 &    0.1        &   0.1   \\\hline
80-90 &    0.1        &   0.1   \\\hline
90-100 &    0.1        &   0.1   \\\hline
100-110 &    0.1        &   0.1   \\\hline
110-120 &    0.1        &   0.1   \\\hline
120-130 &    0.1        &   0.1   \\\hline
130-140 &    0.1        &   0.1   \\\hline
140-160 &    0.1        &   0.1   \\\hline
160-180 &    0.1        &   0.1   \\\hline
180-200 &    0.1        &   0.1   \\\hline
$>$ 200 &    0.1        &   0.1   \\\hline
\hline
\end{tabular}
\caption{Efficiency to pass the cut on \mttwo $>$ 90 \GeV in different channels versus the generated \mttwo.}
\label{tbl:EffMT2}
\end{center}
\end{table}
In the $\ell\Tau$ channels, to calculate the \tauMT, the 4-vector of the visible \Tau and missing particles are used. Table \ref{tbl:EffTauMT}
\begin{table}[!Hhtb]
\begin{center}
\begin{tabular}{|c|c|}
\hline\hline
generated \tauMT (\GeV)  & $\ell\Tau$ \\
\hline\hline
0-50   &   0.1   \\\hline
50-100  &   0.1   \\\hline
100-125  &   0.1   \\\hline
125-150  &   0.1   \\\hline
150-170  &   0.1   \\\hline
170-190  &   0.1   \\\hline
190-200  &   0.1   \\\hline
200-210  &   0.1   \\\hline
210-230  &   0.1   \\\hline
230-250  &   0.1   \\\hline
250-275  &   0.1   \\\hline
275-300  &   0.1   \\\hline
$>$ 300  &   0.1   \\\hline
\hline
\end{tabular}
\caption{Efficiency to pass the cut on \tauMT in $\ell\Tau$ channels versus the generated \tauMT.}
\label{tbl:EffTauMT}
\end{center}
\end{table}
shows the efficiency in the $\ell\Tau$ channels to pass the cut  \tauMT $>$ 200 \GeV versus the generated \tauMT.


In the \tauTau \bintwo, the reconstructed \mttwo is constrained between 40 and 90 GeV. Table \ref{tbl:EffMT2SR2}
\begin{table}[!Hhtb]
\begin{center}
\begin{tabular}{|c|c|}
\hline\hline
generated \mttwo (\GeV)  &  \tauTau \bintwo \\
\hline\hline
0-50   &   0.1   \\\hline
\hline
\end{tabular}
\caption{Efficiency to pass the cut on \mttwo in \tauTau \bintwo versus the generated \mttwo.}
\label{tbl:EffMT2SR2}
\end{center}
\end{table}
shows the efficiency in \tauTau \bintwo to pass the cut 40 $<$ \mttwo $<$ 90 \GeV versus the generated \mttwo. The last selection in this channel is
the cut on \SumMT which is calculated using the 4-vector of two \Tau and missing particles. Table \ref{tbl:EffSumMT} 
\begin{table}[!Hhtb]
\begin{center}
\begin{tabular}{|c|c|c|}
\hline\hline
generated \SumMT (\GeV)  &  \tauTau \bintwo\\
\hline\hline
0-50   &   0.1   \\\hline
\hline
\end{tabular}
\caption{Efficiency to pass the cut on the invariant mass in different channels versus the generated mass.}
\label{tbl:EffSumMT}
\end{center}
\end{table}
shows the efficiency in \tauTau \bintwo to pass the cut \SumMT $>$ 250 \GeV versus the generated \SumMT.

To use these efficiencies, one needs to multiply the values one after another and combine the final value with the values reported in Tab. \ref{tbl:yieldSysSummary} 
statistically, to decide if a signal point is not still excluded. In the generator level, only a pair of $\ell\Tau$ or \tauTau is selected and no other selection is applied.

To take into account the inefficiencies and misidentifications for charge reconstruction of the objects, b-tagging of the jets, identification of the extra leptons 
and the minimum angle between the jets and \MET in the transverse plane, the following extra factors need to be applied.




\section{Conclusions}
\label{sect:conclusion}
A search for SUSY in the $\tau\tau$ final state was performed where the
$\tau$ pair could be produced in a cascade decay from the electroweak production of a chargino pair.  The data analyzed were from proton-proton collisions
%electroweak production of \PSGcpDo pair. in proton-proton collisions 
at $\sqrt{s}$ = 8\TeV collected by the CMS detector at the LHC and corresponding to a\
n integrated luminosity between 18.1 and $19.6~\mathrm{fb}^{-1}$.%, collected by the CMS detector.
To maximize the sensitivity, event selections are optimized for \tauTau (small $\Delta$m), 
\tauTau (large $\Delta$m) and \leptonTau channels using the variables \mttwo, \tauMT, and \SumMT.
The observed number of events is consistent with the SM expectations. 
%We have used different channels and search regions to increase the sensitivity to
%different regions of phase space. All channels considered have at least one hadronic $\tau$ decay.
% \mttwo of two leptons is used as a search variable to
%distinguish between signal and background. In a special part of the phase space with
%a moderate \mttwo, the sum of the transverse mass of two $\tau$ leptons was found to be a
%useful variable.
%There is no excess of events with respect to the SM expectations.
%The other channels were investigated, but they do not add any axclusion power to the analysis.
%Backgrounds and their systematic uncertainties are discussed in details. 
%The expected exclusion limits are also presented for different combination of the channels.
In the context of simplified models, charginos lighter than 421\GeV 
for a massless neutralino are excluded at a 95\% confidence level.
%The upper limits for the direct stau pair production are also provided, 
Upper limits on the direct $\tilde{\tau}\tilde{\tau}$ production cross section are provided, but the limits are more than three times
larger than the theoretical NLO cross sections, 
%region of \sTau mass can not be excluded 
even for a massless neutralino.


% >> acknowledgments (for journal papers)
% Please include the latest version from https://twiki.cern.ch/twiki/bin/viewauth/CMS/Internal/PubAcknow.
%\begin{acknowledgments}...ack-text...\end{acknowledgments}
% This will normally be updated with the text available at the time of submission,
% so please add a comment if it has been modified. Such modifications need to be
% approved.
%

%\begin{acknowledgments}
\section*{Acknowledgements}
\hyphenation{Bundes-ministerium Forschungs-gemeinschaft Forschungs-zentren} We congratulate our colleagues in the CERN accelerator departments for the excellent performance of the LHC and thank the technical and administrative staffs at CERN and at other CMS institutes for their contributions to the success of the CMS effort. In addition, we gratefully acknowledge the computing centers and personnel of the Worldwide LHC Computing Grid for delivering so effectively the computing infrastructure essential to our analyses. Finally, we acknowledge the enduring support for the construction and operation of the LHC and the CMS detector provided by the following funding agencies: the Austrian Federal Ministry of Science, Research and Economy and the Austrian Science Fund; the Belgian Fonds de la Recherche Scientifique, and Fonds voor Wetenschappelijk Onderzoek; the Brazilian Funding Agencies (CNPq, CAPES, FAPERJ, and FAPESP); the Bulgarian Ministry of Education and Science; CERN; the Chinese Academy of Sciences, Ministry of Science and Technology, and National Natural Science Foundation of China; the Colombian Funding Agency (COLCIENCIAS); the Croatian Ministry of Science, Education and Sport, and the Croatian Science Foundation; the Research Promotion Foundation, Cyprus; the Ministry of Education and Research, Estonian Research Council via IUT23-4 and IUT23-6 and European Regional Development Fund, Estonia; the Academy of Finland, Finnish Ministry of Education and Culture, and Helsinki Institute of Physics; the Institut National de Physique Nucl\'eaire et de Physique des Particules~/~CNRS, and Commissariat \`a l'\'Energie Atomique et aux \'Energies Alternatives~/~CEA, France; the Bundesministerium f\"ur Bildung und Forschung, Deutsche Forschungsgemeinschaft, and Helmholtz-Gemeinschaft Deutscher Forschungszentren, Germany; the General Secretariat for Research and Technology, Greece; the National Scientific Research Foundation, and National Innovation Office, Hungary; the Department of Atomic Energy and the Department of Science and Technology, India; the Institute for Studies in Theoretical Physics and Mathematics, Iran; the Science Foundation, Ireland; the Istituto Nazionale di Fisica Nucleare, Italy; the Ministry of Science, ICT and Future Planning, and National Research Foundation (NRF), Republic of Korea; the Lithuanian Academy of Sciences; the Ministry of Education, and University of Malaya (Malaysia); the Mexican Funding Agencies (CINVESTAV, CONACYT, SEP, and UASLP-FAI); the Ministry of Business, Innovation and Employment, New Zealand; the Pakistan Atomic Energy Commission; the Ministry of Science and Higher Education and the National Science Centre, Poland; the Funda\c{c}\~ao para a Ci\^encia e a Tecnologia, Portugal; JINR, Dubna; the Ministry of Education and Science of the Russian Federation, the Federal Agency of Atomic Energy of the Russian Federation, Russian Academy of Sciences, and the Russian Foundation for Basic Research; the Ministry of Education, Science and Technological Development of Serbia; the Secretar\'{\i}a de Estado de Investigaci\'on, Desarrollo e Innovaci\'on and Programa Consolider-Ingenio 2010, Spain; the Swiss Funding Agencies (ETH Board, ETH Zurich, PSI, SNF, UniZH, Canton Zurich, and SER); the Ministry of Science and Technology, Taipei; the Thailand Center of Excellence in Physics, the Institute for the Promotion of Teaching Science and Technology of Thailand, Special Task Force for Activating Research and the National Science and Technology Development Agency of Thailand; the Scientific and Technical Research Council of Turkey, and Turkish Atomic Energy Authority; the National Academy of Sciences of Ukraine, and State Fund for Fundamental Researches, Ukraine; the Science and Technology Facilities Council, UK; the US Department of Energy, and the US National Science Foundation.

Individuals have received support from the Marie-Curie programme and the European Research Council and EPLANET (European Union); the Leventis Foundation; the A. P. Sloan Foundation; the Alexander von Humboldt Foundation; the Belgian Federal Science Policy Office; the Fonds pour la Formation \`a la Recherche dans l'Industrie et dans l'Agriculture (FRIA-Belgium); the Agentschap voor Innovatie door Wetenschap en Technologie (IWT-Belgium); the Ministry of Education, Youth and Sports (MEYS) of the Czech Republic; the Council of Science and Industrial Research, India; the HOMING PLUS programme of Foundation for Polish Science, cofinanced from European Union, Regional Development Fund; the Compagnia di San Paolo (Torino); the Consorzio per la Fisica (Trieste); MIUR project 20108T4XTM (Italy); the Thalis and Aristeia programmes cofinanced by EU-ESF and the Greek NSRF; and the National Priorities Research Program by Qatar National Research Fund. 
%\end{acknowledgments}
%% **DO NOT REMOVE BIBLIOGRAPHY**
\bibliography{auto_generated}   % will be created by the tdr script.

%%% DO NOT ADD \end{document}!

