\section{Introduction}
\label{sect:introduction}
Supersymmetry \cite{Martin:1997ns} (SUSY) is one of the most promising extensions of the 
Standard Model of the elementary particles (SM) which solves both the 
quadratic divergencies and hierarchy problems simultaneously. It introduces a new symmetry between the bosons and fermions and 
for every particle an sparticle is defined which is exactly the same, but differ in spin by 1/2. 

In a hadron collider, like LHC, it is expected to see the signature of the colored SUSY partners, 
but the very extensive search in the LHC experiments pushes the mass of the colored particles much 
beyond the previous expectations. 
Looking at other sectors of the SUSY, e.g, electroweak production of the sparticles, is motivated not to miss SUSY in a corner. 
To cancel the quadratic divergency of the quantum corrections to higgs boson mass,  sleptons, charginos and neutralinos 
should not be heavier than few hundred \GeV \cite{1988NuPhB.306...63B, deCarlos:1993yy}. So they should be accessible by the current data.
A search for new physics using ~20 \fbinv of proton-proton collision data from CMS collected in 2012 in the center of mass energy of 
$\sqrt{s}$ = 8 \TeV  is documented in this report. 
Although the search is sensitive to any high scale new physics with a missing transverse momentum (\MET), 
an R-parity conserving SUSY model is used to illustrate the performance of the method.

Due to the special role of the third generation of the sparticles, events with two taus in the final state 
accompanying with the missing transverse energy (\met) are considered.
The two taus can be generated in the cascade of the staus or charginos:
\begin{linenomath}
\begin{equation}
p + p \rightarrow \tilde{\chi_{1}^{+}} + \tilde{\chi_{1}^{-}} ~~\mathrm{or}~~  p + p \rightarrow \tilde{\tau} + \tilde{\tau}
\end{equation}
\end{linenomath}
when 
\begin{linenomath}
\begin{equation}
\tilde{\chi_{1}^{+}} \rightarrow \tilde{\tau} + \nu ~~\mathrm{or}~~  \tilde{\chi_{1}^{+}} \rightarrow \tilde{\nu}_{\tau} + \tau 
\end{equation}
\end{linenomath}
and 
\begin{linenomath}
\begin{equation}
\tilde{\tau} \rightarrow \tau + \tilde{\chi_{1}^{0}} ~~\mathrm{or}~~  \tilde{\nu}_{\tau} \rightarrow \nu + \tilde{\chi_{1}^{0}} 
\end{equation}
\end{linenomath}
and $\tilde{\chi_{1}^{0}}$ can not be detected and appears as missing transverse momentum (\met).
In this note, we mainly focus on the $\tilde{\chi_{1}^{+}}\tilde{\chi_{1}^{-}}$ production which has a higher 
production cross section. Figure \ref{fig:Productions} shows our favorite decays.

\begin{figure}[!htb]
\centering
\includegraphics[width=0.3\textwidth]{Introductionfigs/DiChargino.png}
\includegraphics[width=0.3\textwidth]{Introductionfigs/DiSTau.png}
\caption{Schematic production of double tau from chargino pair and stau pair.}
\label{fig:Productions}
\end{figure}

The tau leptons can decay to electron or muon in 35\% of the cases (17.5\% for each lepton) 
or can decay via hadrons in 65\% of the cases. Since there are two leptons in the final state, then the probability of having 
two hadronic tau, \tauTau is 42\% and Lepton-\Tau is 46\%. We will see that having less background in \tauTau channel, makes it more powerful
in the final exclusion, although the branching ratio is a little lower.

The search variable which is used to distinguish between the signal and background is the stransverse mass (\mttwo) 
which is the natural extension of the known transverse mass (\mt) to a case 
when two massive particles with equal mass are created in pairs and decay via a chain of jets and leptons to two 
invisible particles. 
In the case of R-Parity conserving SUSY, the Lightest Supersymmetric Particle (LSP) escapes the detection and appears as 
a missing transverse momentum.
The distribution of \mttwo reflects the mass difference between the produced particles and the invisible  particles and is higher for sparticles
compared to the SM particles. Hence, SUSY should appear as an excess in the tail of the \mttwo distribution.
It was shown previously \cite{Chatrchyan:2012jx} that \mttwo is a powerful variable to search for SUSY. 




After introduction in the next section the \mttwo variable is introduced. 




