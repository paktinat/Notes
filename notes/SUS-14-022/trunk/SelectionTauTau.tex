\section{\texorpdfstring{Event selection for the \Tau\Tau channel}{Event selection for the tau-tau channel}}
\label{sect:tauTauCuts}
In this channel events are first selected with a trigger \cite{CMS:2013hoa,Chatrchyan:2012xi,Chatrchyan:2011nv}
that requires the existence of
two loosely identified, isolated \Tau candidates with \PT $>$ 35 \GeV and $|\eta|<$2.1.   Offline,
the \Tau identification requirements are tightened, the two \Tau candidates must be of
opposite charge, and the \PT threshold is increased to 45 GeV.
In events with more than one \Tau\Tau pair, we only consider the pair with the tightest isolated \Tau's. 
%{\bf (What does this mean?  You have two isolation values, so how do you define ``tightest isolation''?)}
%that minimizes the sum of $I_{\Tau}$ 

Events with isolated extra electrons or muons of \PT $>$ 10 \GeV and $|\eta| <$ 2.4 
are rejected to suppress %the contribution of the associatiated production of \Z with vector bosons.
backgrounds from diboson decays.
%A Monte Carlo (MC) study shows that the invariant mass distribution of the two \Tau's coming from a \Z boson can be approximated 
%with a gaussian distribution with a peak close to 70 \GeV and a width of about 15 \GeV, so the events with the invariant mass of  
%the two \Tau's in this range are excluded (\Z-veto). 
The background from \Z$\rightarrow$\Tau\Tau events is reduced by rejecting events where the visible
di-\Tau invariant mass is between 55 and 85 \GeV.  Furthermore, events where this invariant mass is below 15 GeV
are also rejected to eliminate low mass QCD resonances.
The minimum angle $\Delta \Phi$
in the transverse plane between the $\VEtmiss$ and any of the jets %with \PT $>$ 40 \GeV and $|\eta| <$ 5 
must be greater than 1.  This requirement reduces backgrounds from QCD multijet events and \wjets events.
We also impose a loose requirement \mttwo $>$ 40 \GeV to increase the rejection power against QCD multijet events.


After applying the pre-selection described above,
extra requirements are introduced to define two search regions.
The first search region is designed to be sensitive to models
with large mass difference ($\Delta M$) between charginos and neutralinos.
In this case, the \mttwo signal distribution extends beyond the 
distribution of SM backgrounds.
The second search region is dedicated to models with small $\Delta M$.  In this case
the sum of the \mt of two \Tau, \SumMT = $\mt^{\Tau^1} + \mt^{\Tau^2}$, is a good discriminator
between signal and SM backgrounds. 
This variable was used previously in a similar analysis from the ATLAS experiment \cite{Aad:2014yka}.


%When the mass difference is sufficiently high, \mttwo has a tail well beyond 80 \GeV, which is 
%the peak  and the end point of the \mt distribution of \wjets events in the parton level. 
%Due to the particles' width and the detector effects, 
%the endpoint is moved to the higher values and the optimized cut for this variable is 90 \GeV. 

%If the mass diffrenece is not too high, %\mttwo can not exceed 80 \GeV, but 
%the sum of the \mt of two \Tau, \SumMT = $\mt^{\Tau^1} + \mt^{\Tau^2}$, can be a good descriminator between the signal and SM backgrounds. 
The two signal regions (SR) are defined as:
\begin{itemize}
\item {\bf SR1}: \mttwo $>$ 90 \GeV;
\item {\bf SR2}: \mttwo $<$ 90 \GeV and \SumMT $>$ 250 \GeV; b-tagged jets are vetoed.
\end{itemize}
The veto on b-tagged jets in SR2 is intended to reduce
the background from $t\bar{t}$, which
tends to populate the low \mttwo region.  For this purpose, jets are btagged
using the medium working point of the combined secondary vertex algorithm \cite{Chatrchyan:2012jua}. 


%The b-jets have \PT $>$ 20 \GeV and $|\eta| < $ 2.4.

%The event yields can be found in table~\ref{tbl:cutflowtable}. The yields for three SUSY signal points, 
%corresponding to a low mass difference $(m_{\chione}=180\GeV,~m_{\PSGczDo}=60\GeV)$, 
%a moderate mass difference $(m_{\chione}=240\GeV,~m_{\PSGczDo}=40\GeV)$ and a high mass difference $(m_{\chione}=380\GeV,~m_{\PSGczDo}=1\GeV)$, are presented in the table.   
%\begin{sidewaystable}
%\begin{table}[!Hhtb]
%\begin{center}
%\begin{small}
%\begin{tabular}{lcccccccccc}
%\hline\hline
%  &SUSY(180,60)&(240,40)&(380,1)&Higgs&QCD&WW&W&DY&Top&Total Bkg\\%&Data\\
%\hline\hline
%\multirow{5}{*}{Pre-Selection}&2 $\tau_h$ Selection&41.97&30.96&11.28&87.67&22081.57&13.71&595.80&2133.23&115.33&25027.32$\pm$6971.15&19615\\
%&$e$ and $\mu$ Veto&38.68&27.89&9.87&81.53&19272.05&11.21&543.42&1961.29&95.85&21965.34$\pm$6387.87&18526\\
%&Z Veo&37.80&26.28&9.21&70.50&18825.02&10.86&527.83&1333.37&88.53&20856.11$\pm$6383.93&17554\\
%&$\mindphifour > $ 1&17.95&15.16&6.13&13.91&8426.98&3.66&192.11&276.27&13.67&8926.59$\pm$4404.31&5105\\
%&$M_{T2} > $ 40&9.50&11.66&5.60&0.89&135.29&1.11&31.93&13.17&5.26&187.65$\pm$135.47&131\\
%\hline
%\binone&$M_{T2} > $ 90&0.59&3.89&3.81&0.17&$<$135.29&0.02&$<$1.28&0.56&$<$0.47&0.75$\pm$0.08&1\\
%\binone &0.59&3.89&3.81&0.17& 0.0 &0.02&0.0&0.56&0.0 &0.75$\pm$0.08\\%&1\\
%\hline
%\multirow{3}{*}{\bintwo}&b-jet veto&7.92 &9.33 &4.67 &0.75&135.20&0.96&29.13&11.15&0.78&177.98$\pm$135.36&115\\
%&$M_{T2} < $ 90&7.42 &6.17 &1.51 &0.61&135.20&0.94&29.13&10.65&0.78&177.32$\pm$135.36&114\\
%\bintwo &2.17&3.36  &1.08&0.07& 0.0 &0.15&0.43&0.81&0.53&1.99$\pm$0.87\\%&2\\
%\hline\hline
%\end{tabular}
%\caption{ MC driven yields for $\tauTau$ channel. The quoted uncertainties are only statistical.}
%\label{tbl:cutflowtable}
%\end{small}
%\end{center}
%\end{table}
%\end{sidewaystable}

The distributions of $\mttwo$ and $\SumMT$ for data and the SM prediction
are shown in Fig.\ref{fig:comparison} before the application of the final
requirements listed above.
%{\bf (Is the b-veto applied?  This needs to be explained.)}
In the $\SumMT$ distribution, the b-jet veto and \mttwo $<$ 90 \GeV are also applied.
The SM predictions in Fig.~\ref{fig:comparison} 
are from Monte Carlo, except for the
QCD prediction which is taken from same-sign di-tau data events,
after subtracting a small contribution of same-sign non-QCD events estimated from Monte Carlo.
The data and SM predictions are in good agreement within the statistical uncertainties.
The expected distributions for a SUSY signal 
corresponding to a moderate mass difference $(m_{\chione}=240\GeV,~m_{\PSGczDo}=40\GeV)$
is also shown for illustration puroposes.
\begin{figure}[!Hhtb]
\centering
%\includegraphics[angle=0,scale=0.35]{TauTauFigs/MT2_SSQCD_dataunblinding.png}
\includegraphics[angle=0,scale=0.35]{TauTauFigs/mt2.png}
\includegraphics[angle=0,scale=0.35]{TauTauFigs/summt.png} \\ 
%\includegraphics[angle=0,scale=0.35]{TauTauFigs/SumMT_SSQCD_dataunblinding.png} \\
\caption{The distributions of \mttwo (left) and \SumMT (right) after applying all requirements. 
The last bins of the histograms correspond to the two signal regions (SR1 and SR2). }
\label{fig:comparison}
\end{figure}
