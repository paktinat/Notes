\section{Selection Cuts for TauTau Channel}
\label{sect:tauTauCuts}
The selection of the events from the collisions is done with the dedicated triggers that use the combination of muon, electron 
and tau lepton reconstructed in the trigger levels ~\cite{CMS:2013hoa,Chatrchyan:2012xi,Chatrchyan:2011nv}.

To select the events for \Tau\Tau channel, two reconstructed \Tau which are oppsitlely charged are selected. Both taus are asked to 
have \PT $>$ 45 GeV/$c$ and $|\eta| <$ 2.1 and $I_{\Tau}$ less than XX GeV. In the case of more than one pair, the pair that minimizes the sum of
$I_{\Tau}$ is selected.
The events with extra electrons or muons (\PT $>$ 10 GeV, $\eta >$ 2.1 and loose isolation criteria) 
are rejected to suppress the association production of Z with vector bosons.

A monte carlo study shows that the invariant mass distribution of two \Tau coming from a $Z$ boson can be approximated with a gaussian 
distribution with a peak close to 70 GeV/$c^2$ and a width of about 15 GeV/$c^2$, so the events with the invariant mass of  
two \Tau in this range are excluded ($Z$-veto). A minimum cut of 15 GeV/$c^2$ on this invariant mass helps to get rid of the low mass
resonances. The minimum angle in the transverse plane between the \MET and the jets with \PT $>$ 40 GeV and $|\eta| <$ 5.0 
is asked to be greater than 1.0. This is a cut against the QCD multijet events and every event with a fake \MET. A soft cut on 
\mttwo (40 GeV) can increase this rejection.

After applying pre-selection cuts, two extra set of cuts are introduced, one aims the regions with large and medium mass difference between 
charginos and neutralinos, the other one is dedicated for low mass regions. The two search bins are chosen to be exclusive in order 
to be able to combine the statistics at the end. When the mass difference is sufficiently high, \mttwo has a tail well beyond 80 GeV, which is 
the peak  and the end point of the \mt distribution of $W$jets events. If the mass diffrenece is not too high, \mttwo can not exceed 80 GeV, 
but the sum of the \mt of two \Tau (\SumMT = $\mt^{\Tau^1} + \mt^{\Tau^2}$) can be a good descriminator between the signal and SM backgrounds. 
Two bins are defined as:
\begin{enumerate}
\item \mttwo $>$ 90 \GeV.
\item \mttwo $<$ 90 \GeV; \SumMT $>$ 250 \GeV; b-tagged jets are vetoed.
\end{enumerate}
Vetoing the b-tagged jets in the second bin can be useful against the backgrounds containting a top quark, especially \ttbar events. 


Theevent yields can be found in table~\ref{tbl:cutflowtable}. The yields for three SUSY signal points, corresponding to a low mass difference $(m_{\chione}=180\GeV,m_{\PSGczDo}=60\GeV)$, a moderate mass difference $(m_{\chione}=240\GeV,m_{\PSGczDo}=40\GeV)$ and a high mass difference $(m_{\chione}=380\GeV,m_{\PSGczDo}=1\GeV)$, are presented in the table.   
%\begin{sidewaystable}
\begin{table}[!Hhtb]
\begin{center}
\begin{small}
\begin{tabular}{lccccccccccc}
\hline\hline
  &SUSY(180,60)&(240,40)&(380,1)&Higgs&QCD&WW&W&DY&Top&Total Bkg&Data\\
\hline\hline
%\multirow{5}{*}{Pre-Selection}&2 $\tau_h$ Selection&41.97&30.96&11.28&87.67&22081.57&13.71&595.80&2133.23&115.33&25027.32$\pm$6971.15&19615.00\\
%&$e$ and $\mu$ Veto&38.68&27.89&9.87&81.53&19272.05&11.21&543.42&1961.29&95.85&21965.34$\pm$6387.87&18526.00\\
%&Z Veo&37.80&26.28&9.21&70.50&18825.02&10.86&527.83&1333.37&88.53&20856.11$\pm$6383.93&17554.00\\
%&$\mindphifour > $ 1&17.95&15.16&6.13&13.91&8426.98&3.66&192.11&276.27&13.67&8926.59$\pm$4404.31&5105.00\\
%&$M_{T2} > $ 40&9.50&11.66&5.60&0.89&135.29&1.11&31.93&13.17&5.26&187.65$\pm$135.47&131.00\\
%\hline
%\binone&$M_{T2} > $ 90&0.59&3.89&3.81&0.17&$<$135.29&0.02&$<$1.28&0.56&$<$0.47&0.75$\pm$0.08&1.00\\
\binone &0.59&3.89&3.81&0.17&$<$135.29&0.02&$<$1.28&0.56&$<$0.47&0.75$\pm$0.08&1.00\\
\hline
%\multirow{3}{*}{\bintwo}&b-jet veto&7.92 &9.33 &4.67 &0.75&135.20&0.96&29.13&11.15&0.78&177.98$\pm$135.36&115.00\\
%&$M_{T2} < $ 90&7.42 &6.17 &1.51 &0.61&135.20&0.94&29.13&10.65&0.78&177.32$\pm$135.36&114.00\\
\bintwo &2.17&3.36  &1.08&0.07&$<$135.20&0.15&0.43&0.81&0.53&1.99$\pm$0.87&2.00\\
\hline\hline
\end{tabular}
\caption{Yields for $\tauTau$ channel. The quoted uncertainties are only statistical. When the remaining events from MC are zero, the weight of the events is reported as the upper bound.}
\label{tbl:cutflowtable}
\end{small}
\end{center}
\end{table}
%\end{sidewaystable}
