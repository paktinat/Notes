\section{Results and interpretation}
\label{sect:stat}

\begin{table}[!htb]
\begin{center}
\begin{small}
\caption{Data yields and background predictions with uncertainties in the four signal regions of the search. 
The uncertainties are reported in two parts, the statistical and systematic uncertainties, respectively. 
The \wjets and QCD multijet main backgrounds are derived from data as described in Section~\ref{sect:bkg}; 
the abbreviation ``VV'' refers to diboson events. The yields for three signal points representing the low, medium, and high $\Delta m$
are also shown. SUSY(X, Y) stands for a SUSY signal with ${\rm{m}}_{\chione}$ = X\GeV and ${\rm{m}}_{\neutralino}$ = Y\GeV.}
\begin{tabular}{|c|c|c|c|c|}
\hline
	           & \eTau & \muTau & \tauTau \binone & \tauTau \bintwo \\
\hline
  DY               & 0.19 $\pm$ 0.04 $\pm$ 0.03 & 0.25 $\pm$ 0.06  $\pm$ 0.04  &  0.56 $\pm$ 0.07 $\pm$ 0.12 & 0.81 $\pm$ 0.56 $\pm$ 0.18  \\
tX, VV, hX  & 0.03 $\pm$ 0.03 $\pm$ 0.02 & 0.19 $\pm$ 0.09  $\pm$ 0.09  &  0.19 $\pm$ 0.03 $\pm$ 0.09 & 0.75 $\pm$ 0.35 $\pm$ 0.38  \\
\wjets             & 3.30$_{- 3.30}^{+ 3.35}$ $\pm$ 0.56 & 8.15 $\pm$ 4.59  $\pm$ 1.53  &  0.70 $\pm$ 0.21 $\pm$ 0.55 & 4.36 $\pm$ 1.05 $\pm$ 1.63  \\
QCD multijet       &             -              &            -                 &  0.13 $\pm$ 0.06 $\pm$ 0.21 & 1.15 $\pm$ 0.39 $\pm$ 0.74  \\
\hline
SM total           & 3.52 $\pm$ 3.35 $\pm$ 0.56 & 8.59 $\pm$ 4.59  $\pm$ 1.53  &  1.58 $\pm$ 0.23 $\pm$ 0.61 & 7.07 $\pm$ 1.30 $\pm$ 1.84  \\
\hline
Observed           &               3            &                5             &             1               & 2     \\\hline  
SUSY(380, 1)        & 2.14 $\pm$ 0.08 $\pm$ 0.38 & 2.16 $\pm$ 0.08  $\pm$ 0.39  &  4.10 $\pm$ 0.10 $\pm$ 0.90 & 1.10 $\pm$ 0.05 $\pm$ 0.27 \\
SUSY(240, 40)       & 1.43 $\pm$ 0.19 $\pm$ 0.21 & 0.96 $\pm$ 0.14  $\pm$ 0.14  &  4.35 $\pm$ 0.27 $\pm$ 0.91 & 3.60 $\pm$ 0.25 $\pm$ 0.83 \\
SUSY(180, 60)       & 0.12 $\pm$ 0.04 $\pm$ 0.02 & 0.04 $\pm$ 0.02  $\pm$ 0.01  &  0.73 $\pm$ 0.11 $\pm$ 0.17 & 2.36 $\pm$ 0.17 $\pm$ 0.54 \\
\hline
\end{tabular}
\label{tbl:yieldSysSummary}
\end{small}
\end{center}
\end{table}


Event data yields and background predictions for the four signal regions are summarized in Table~\ref{tbl:yieldSysSummary}.
There is no excess of events over the SM expectation.  We interpret our results in the context
of a simplified model of chargino pair-production and decay, which corresponds to the left
diagram in Fig.~\ref{fig:Productions}.  Combining all four signal regions,
the search rules out \chione of mass up to 410 \GeV and  $\PSGczDo$ of mass up to 100 \GeV,
see Fig.~\ref{fig:limit_final}.

{\bf (You need to spend one sentence to explain what method you are using to set limits.  I assume
  it is CLs, but because I am not sure what you actually did I did not write
  it down.  Also, you
  need to specify whether this is 95\% CL or 90\% CL or whatever).}

{\bf (The description of the simplified model is insufficient.  Surely the cross-section and branching
  ratios depends on the stau mass and/or the sneutrino mass?  It must also make a difference whether
  the stau/sneutrino are on- or off-shell and what their masses are as far as the acceptance
  is concerned.
  Also in Figure 1 left there are two distinct
  decay chains.  What BR are you assuming for the two chains)}

{\bf The caption for the figure is incorrect..it says expected but there is also observed.  The boundaries
  of the plot (diagonal lines) need to be explained.)}

%%%%%%%%%%
\begin{linenomath}
\begin{figure}[h]
\centering
\includegraphics[width=0.7\textwidth,keepaspectratio=true]{StatisticsFig/Exclusion4Bins.pdf}
\caption{Expected exclusion power in terms of Simplified Models
with the total dataset of 2012. 
%Backgrounds are predicted using Monte-Carlo simulations and a rough estimate of systematic uncertainties equal $10\%$ is taken into account.
}
\label{fig:limit_final}
\end{figure}
\end{linenomath}
%%%%%%%%%%



