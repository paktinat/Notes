\subsection{\texorpdfstring{$\tauTau$ channel}{tau-tau channel}}
\label{sect:tauTauCuts}
In the final state with 2 hadronic taus, events are required to pass the following list of cuts
\begin{itemize}
\item \textbf{2 $\boldsymbol\tau_h$ Selection} At least two reconstructed opposite-sign tau objects, 
with $\pt>45\GeV$ and $|\eta|<2.1$, are required. Since the online trigger cuts require two taus with 
$\pt>35\GeV$, there is an offline cut on $\pt>45\GeV$ to get closer to the plateau of the trigger efficiency. 
Both taus not only are required to pass \emph{MediumCombinedIsoDBSumPtCorr3Hits} isolation requirement, 
but also should pass the Loose working point of the MVA3 anti-e discriminator, being \emph{LooseElectronMVA3Rejection}, 
as well as the Loose working point of the Muon2 anti-$\mu$ discriminator, being \emph{LooseMuon2Rejection}. 
In order to reject low mass QCD resonances, the invariant mass of two taus is required to be above 15 \GeV. Also a loose cut on \MPT $>$ 30 \GeV is applied to suppress QCD events.
\item \textbf{\MPT $>$ 30 \GeV} A soft cut is applied on the missing transverse momentum to suppress the QCD multijet events. 
The \MPT related variables which are introduced later, need a minimum value for \MPT to be meaningful. 
\item \textbf{$\boldsymbol e$ and $\boldsymbol\mu$ Veto} Extra electrons and muons are vetoed. This 
cut would help to reject backgrounds from di-boson events. Electron and muon rejection 
are exactly the same as in $\mu\Tau$ channel which is described in Sec.~\ref{sect:leptonTauCuts}.
\item \textbf{Z Veto} Events under the reconstructed Z mass peak are rejected, namely $|m_{\tau\tau}-m_{\Z}^{\rm rec}|>15\GeV$. 
In the reconstructed $\Z\rightarrow \Tau\Tau$ events passing above-mentioned set of cuts, events under the Z mass peak are fitted 
with a Gaussian function. The mean value of the fit is found to be about 70 \GeV. Hence the invariant mass of two taus, $m_{\tau\tau}$, should be 
either below 55 \GeV or above 85 \GeV.  
\item \textbf{$\boldsymbol\mindphifour > $ 1} The minimum angle in the transverse plane between the \MPT and the jets with \PT $>$ 40 \GeV and $|\eta| <$ 5 is asked to be greater than 1. This cut is introduced against multi-jet events.
\item \textbf{$\boldsymbol\mttwo>40\GeV$} Since SM background events with low \MPT or fake \MPT arising from miss-measured reconstructed object
 contribute to the lower values of $\mttwo$, then a moderate cut on this variable would improve data over MC ratio.  
\end{itemize}
After applying preselection cuts, two extra set of cuts are introduced, where each of them is sensitive to some regions of $\chione\PSGczDo$ mass plane: one aims the regions with large mass difference between charginos and neutralinos, the other one is dedicated for low mass-difference regions. It should be noted that the two signal regions ({\bf SR}) are chosen to be exclusive in order to be able to combine the statistics at the end.
\begin{itemize}
\item {\bf \binone: sensitive to high $\boldsymbol\Delta m$}
A first set of cuts includes a cut on $\mttwo$.
\begin{itemize}
\item $\mttwo>90\GeV$.
\end{itemize}
\item {\bf \bintwo: sensitive to low $\boldsymbol\Delta m$}
A second set of cuts includes some cuts on different variables listed below.
\begin{itemize}
\item bjets tagged with CSVM algorithm are vetoed.
\item $\mttwo<90\GeV$;
\item $\SumMT>250\GeV$;
\end{itemize}
\end{itemize}
\SumMT = $\mt^{\Tau^1} + \mt^{\Tau^2}$, is a good discriminator
between signal and SM backgrounds in \bintwo. 
This variable was used previously in a similar analysis from the ATLAS experiment \cite{Aad:2014yka}.

The cut-flow-table can be found in table~\ref{tbl:cutflowtable}. The yields for three SUSY signal points, corresponding to a low mass difference $(m_{\chione}=180\GeV,m_{\PSGczDo}=60\GeV)$, a moderate mass difference $(m_{\chione}=240\GeV,m_{\PSGczDo}=40\GeV)$ and a high mass difference $(m_{\chione}=380\GeV,m_{\PSGczDo}=1\GeV)$, are presented in the table.   
\begin{sidewaystable}
%\begin{table}
\begin{center}
\begin{small}
\caption{Cut-flow-table for $\tauTau$ channel. The quoted uncertainties are only statistical. It should be noted that zero 
QCD events are left in MC sample with \mttwo $>$ 40 \GeV. The data-driven estimate of the QCD events is given in Section \ref{sect:bkgQCD}, showing 68.5 events are expected in this step.}
\begin{tabular}{llccccccccccc}
\hline\hline
&  &SUSY(180,60)&(240,40)&(380,1)&Higgs&QCD&WW&W&DY&Top&Total Bkg&Data\\
\hline\hline
%\multirow{5}{*}{Pre-Selection}&2 $\tau_h$ Selection&41.97&30.96&11.28&87.67&22081.57&13.71&595.80&2133.23&115.33&25027.32$\pm$6971.15&19615\\
%&$e$ and $\mu$ Veto&38.68&27.89&9.87&81.53&19272.05&11.21&543.42&1961.29&95.85&21965.34$\pm$6387.87&18526\\
%&Z Veo&37.80&26.28&9.21&70.50&18825.02&10.86&527.83&1333.37&88.53&20856.11$\pm$6383.93&17554\\
%&$\mindphifour > $ 1&17.95&15.16&6.13&13.91&8426.98&3.66&192.11&276.27&13.67&8926.59$\pm$4404.31&5105\\
%&$M_{T2} > $ 40&9.50&11.66&5.60&0.89&135.29&1.11&31.93&13.17&5.26&187.65$\pm$135.47&131\\
\multirow{5}{*}{Pre-Selection}&\MPT, 2 $\tau_h$ Selection& 36.09 &30.92& 11.68 &45.55&5671.12&10.54&375.40&994.02&96.36&7192.98$\pm$3411.96&6991.00\\
&$e$ and $\mu$ Veto& 33.05 &27.97& 10.19 &41.37&2861.59&8.57&341.25&908.39&79.38&4240.56$\pm$1962.04&6612.00\\
&Z Veo& 32.27 &26.31& 9.50 &31.30&2734.04&8.26&328.54&573.64&73.58&3749.36$\pm$1957.86&6109.00\\
&$\mindphifour > $ 1& 15.50 &15.75& 6.38 &4.50&277.39&2.62&118.09&68.51&11.99&483.10$\pm$197.08&1544.00\\
&$M_{T2} > $ 40& 10.04 &13.13& 5.96 &0.78&0.00&1.06&31.93&9.19&5.26&48.23$\pm$6.82&111.00\\
\hline
%\binone&$M_{T2} > $ 90&0.59&3.89&3.81&0.17&$<$135.29&0.02&$<$1.28&0.56&$<$0.47&0.75$\pm$0.08&1\\
\binone&$M_{T2} > $ 90& 0.73 &4.35& 4.10 &0.17&0.00&0.02&0.00&0.56&0.00&0.75$\pm$0.08&1.00\\
\hline
%\multirow{3}{*}{\bintwo}&b-jet veto&7.92 &9.33 &4.67 &0.75&135.20&0.96&29.13&11.15&0.78&177.98$\pm$135.36&115\\
%&$M_{T2} < $ 90&7.42 &6.17 &1.51 &0.61&135.20&0.94&29.13&10.65&0.78&177.32$\pm$135.36&114\\
%&$\Sigma M_T^{\tau_i} > $ 250&2.17&3.36  &1.08&0.07&$<$135.20&0.15&0.43&0.81&0.53&1.99$\pm$0.87&2\\
\multirow{3}{*}{\bintwo}&b-jet veto& 8.37 &10.28& 4.95 &0.68&0.00&0.91&29.13&8.22&0.78&39.73$\pm$6.48&95.00\\
&$M_{T2} < $ 90& 7.74 &6.77& 1.57 &0.54&0.00&0.89&29.13&7.72&0.78&39.07$\pm$6.48&94.00\\
&$\Sigma M_T^{\tau_i} > $ 250& 2.36 &3.60& 1.10 &0.07&0.00&0.15&0.43&0.81&0.53&1.99$\pm$0.87&2.00\\
\hline\hline
\end{tabular}
\label{tbl:cutflowtable}
\end{small}
\end{center}
%\end{table}
\end{sidewaystable}
The distributions of $\mttwo$ and $\SumMT$ just before applying the final cut on the search variable 
are shown in figure~\ref{fig:comparison}. 
%The b-jet veto cut is also applied on the $\SumMT$ distribution. 
For both plots, the shape of QCD MC events 
is taken from same-sign events, which means that same selection cuts as pre-selection are applied except for the charge 
requirement which is reversed. Then the distribution of QCD MC events are found from data after non-QCD MC subtraction 
in the same-sign region. It can be concluded that data and MC are in reasonable agreement within the statistical uncertainties. 
\begin{figure}[!Hhtb]
\centering
\includegraphics[angle=0,scale=0.375]{TauTauFigs/MT2_SSQCD.png}
\includegraphics[angle=0,scale=0.375]{TauTauFigs/SumMT_SSQCD.png} \\
\caption{The distributions of \mttwo (left) and \SumMT (right) after applying all cuts. The last bins are the signal regions.}
\label{fig:comparison}
\end{figure}
