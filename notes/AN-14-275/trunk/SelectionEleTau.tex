\section{Selection Cuts for eleTau Channel}
\label{sect:eleTauCuts}
In lepton (electron and muon) + hadronic $\tau$ channels, tight isolated $\hadtau$ objects are selected. To reduce the rate of the fake $\hadtau$'s originated from electron and $\mu$, the same cuts on the discriminators against electron and $\mu$ which are used in the Higgs search analysis have been applied. In electron + $\hadtau$ channel, $\hadtau$ objects which pass MediumMVA3Rejection against electron and the LooseMuon2RejectionPF discriminator against $\mu$ are selected. In the $\mu + \hadtau$ channel, the tight working point of the Muon2Rejection is requested and the Loose working point of the discriminator against electron is applied.

 
In electron + $\hadtau$ channel, each event is requested to have an electron with $\pt>20\GeV$ in the $|\eta| < 2.1 $ region. A tight cut of $0.1$ on the isolation and $0.1$ on the $dZ$ of the selected electron are also applied.

To suppress dilepton and multilepton backgrounds, events with and extra electron or $\mu$ with $\pt>10\GeV$ are rejected. For the extra electron, a wider window of $|\eta|<2.3$ is scanned and a looser isolation cut of $0.2$ is applied. For the extra $\mu$, a selection similar to the $\mu$ selection in $\mu+\hadtau$ channel is applied.


After requesting two opposite sign leptons in the events, a loose cut on \MET $>30\GeV$ is applied to suppress QCD events. As there is no b-quark in the signal, rejecting events with one or more b-tagged jets helps a lot in reducing $\ttbar$ and W+b backgrounds.

To reject QCD low mass resonances, the invariant mass of the lepton and the hadronic $\hadtau$ is requested to be greater than $15\GeV$. Another cut on the invariant mass of the di-lepton system is applied to remove the peak of the Z+jets events. It has been found that the visible mass of the $\Z\to\hadtau\hadtau\to\,e +\,\hadtau$ moves to $60 \pm 15\GeV$ due to mis-reconstruction of the energy of the hadronic $\hadtau$ and also the missing energy due to the decay of the $\hadtau$ to electron. So the window of $45 < m_{e\hadtau} < 75$ is cutted away. As the last pre-selection cut, events with $\mttwo<40\GeV$ are discarded to kill the bulk of the QCD events and get rid of related uncertainities due to mis-reconstruction of the QCD events. As it has mentioned above, the signal events are expected to have high $\mttwo$ values and are not removed with such a cut.

The cut flow tables for the $e/\mu$ $\,+\hadtau$ pre-selections are shown in tables .... and ... respectively. The distribution of the $\pt$ of the $\hadtau$ and $\MET$ in the pre-selected events in both channels are shown in FIGS ... . The good agreement between data and MC confirms that the needed correction factors are considrred carefully.

Similar to the di-hadronic $\hadtau$ channel, first we find the optimized cut on the $\mttwo$ to supress backgrounds especially W+jet events. As it has been shown in FIG ..., the best value to cut on, similar to the $\hadtau\hadtau$ channel is $\mttwo > 90\GeV$ for both $e/\mu+\hadtau$ channels. Such a high cut on the $\mttwo$ increases the sensitivity of the study to signal events with high $\Delta m(\chione,\PSGczDo)$ values. We then investigate the shape of different variables for signal and backgrounds and try to find the most optimized cut to have the best exclusion. The most sensitive variable for both channels are found to be the $\hadtau$ transverse mass. As it has been shown in FIG ..., the best cut value for the high mass difference signal is $\mt_{,\hadtau} > 200\GeV$. You can find the composition of the backgrounds and number of remaining signals for bothe channels in tables ... and ... respectively.

Opposite to the $\hadtau\hadtau$ channel, the events with $\mttwo<90\GeV$ are not useful in $e$/$\mu + \hadtau$ channels because of the contamination of the W+jets events.
