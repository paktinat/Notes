\section{Introduction}
\label{sect:introduction}

Supersymmetry (SUSY) \cite{Golfand:1971iw,Wess:1973kz,Wess:1974tw,Fayet1,Fayet2} is one of the most promising extensions of the 
standard model (SM) of elementary particles.  
Certain classes of SUSY models can lead to the unification of gauge couplings at high energy, 
provide a solution to the gauge hierarchy problem without fine tuning by stabilizing the mass of the Higgs boson 
against large radiative corrections, and provide a stable dark matter candidate in models with conservation of R-parity.
A key prediction of SUSY is the existence of new particles with the same gauge quantum numbers as SM particles but
differing by a half-unit in spin (sparticles).


Extensive searches at the LHC have excluded the existence of strongly produced (colored) sparticles in a broad range of scenarios, 
with lower limits on sparticle masses ranging up to 1.8 TeV for gluino pair production 
\cite{Chatrchyan:2013fea,Chatrchyan:2013mys,Chatrchyan:2014aea,Chatrchyan:2014lfa,Khachatryan:2015vra,Khachatryan:2015lwa,Aad:2015pfx,Aad:2015iea}. 
While the limits do depend on the details of the assumed SUSY particle mass spectrum, 
constraints on the colorless sparticles are generally much less stringent.
This motivates the electroweak SUSY search described in this paper.


Searches for charginos ($\widetilde{\chi}^{\pm}\xspace$), neutralinos ($\widetilde{\chi}^{0}\xspace$), and sleptons ($\widetilde{\ell}\xspace$) by the ATLAS and CMS Collaborations are described in Refs.~\cite{Aad:2014nua,Aad:2014vma,Khachatryan:2014qwa,Khachatryan:2014mma,Khachatryan:2015kxa}.
In various SUSY models, the lightest of the SUSY partners of the SM leptons are those of the third generation, 
resulting in enhanced branching fractions for final states with $\tau$ leptons~\cite{Martin:1997ns}.  
The previous searches for charginos, neutralinos, and sleptons by the CMS Collaboration  either did not include the possibility that 
the scalar $\tau$ lepton and its neutral partner (\stau and $\sNu_\tau$) 
are the lightest sleptons \cite{Khachatryan:2014qwa}, or that the initial charginos and neutralinos are produced in vector-boson fusion processes \cite{Khachatryan:2015kxa}. An ATLAS search for SUSY in the di-$\tau$ channel is reported in Ref.~\cite{Aad:2014yka}, excluding chargino masses up to 345\GeV 
for a massless neutralino (\PSGczDo).

In this paper, a search for the electroweak production of the lightest charginos (\chione) and scalar $\tau$  leptons (\stau) is reported using events 
with two opposite-sign $\tau$ leptons and 
missing transverse momentum (\MPT), assuming the masses of the third-generation sleptons are between those of the 
chargino and the lightest neutralino. 
Two $\tau$ leptons can be generated in the decay chain of \chione and \sTau, as shown in Fig.~\ref{fig:Productions}. 
\begin{figure}[!htb]
\centering
\includegraphics[width=0.45\textwidth]{Introductionfigs/TChipmSlepSnu.pdf}
\includegraphics[width=0.41\textwidth]{Introductionfigs/TSlepSlep.pdf}

\caption{Schematic production of $\tau$ lepton pairs from chargino (left) or $\tau$ slepton (right) pair production.}
\label{fig:Productions}
\end{figure}
%Figure \ref{fig:Productions} shows a representative diagram of pair production of charginos in such a final state. 
The results of the search are interpreted in the context of SUSY simplified model spectra (SMS) \cite{Alwall:2008ag,alves:sms} for both
production mechanisms.


The results are based on a data set of proton-proton (pp)
collisions at $\sqrt{s}$ = 8\TeV
collected with the CMS detector at the LHC during 2012, corresponding to integrated
luminosities of 18.1 and 19.6 \invfb in different channels. 
This search makes use of the stransverse mass variable (\mttwo)~\cite{Lester:1999tx,Barr:2003rg},
which is the extension of transverse mass (\mt) to the case 
where two massive particles with equal mass are created in pairs  
and decay to two invisible and two visible particles. 
In the case of this search, the visible particles are both $\tau$ leptons.
The distribution of \mttwo reflects the scale of the produced particles and has a longer tail for heavy sparticles
compared to lighter SM particles. Hence, SUSY 
can manifest itself
as an excess of events in the high-side tail of the \mttwo distribution. 
Final states are considered where
two $\tau$ leptons are each reconstructed via hadronic decays (\tauTau), 
or where only one $\tau$ lepton  decays hadronically and 
the other decays leptonically (\leptonTau, where $\ell$ is an electron or muon). 

The paper is organized as follows.  The CMS detector, the event reconstruction, and the data sets are described
in Sections \ref{sect:CMSRec} and \ref{sect:MCSamples}. The \mttwo variable is introduced in Section \ref{sect:mt2def}. 
The selection criteria for the \tauTau and \leptonTau channels are described in Section \ref{sect:tauTauCuts} and \ref{sect:eleTauCuts}, respectively. 
A detailed study of the SM backgrounds is presented in Section \ref{sect:bkg}, while Section \ref{sect:sys} 
is devoted to the description of the systematic uncertainties.  The results of the search with its statistical interpretation are presented in 
Section \ref{sect:stat}. Summary is given in Section \ref{sect:conclusion}. The efficiencies for the important selection criteria are summarized in Appendix \ref{sect:model} and can be used to interpret these results within other phenomenological models. 




