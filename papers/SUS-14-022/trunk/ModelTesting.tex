\section{Additional information for new model testing} 
\label{sect:model}
In the previous sections, a simplified SUSY model is used to optimize the selection criteria and interpret the results. 
Here, the main efficiencies versus generated values are reported, so that these results can be used in 
an approximate manner to examine new models in a MC generator-level study. 
The number of the passed signal events and its uncertainty that 
can be evaluated by a generator-level study 
should be combined statistically with the results in Table \ref{tbl:yieldSysSummary} to find the upper limit 
on the number of signal events
and decide if a model is excluded or still allowed according to  the analysis presented in this paper.

Efficiencies are provided as a function of the kinematic properties (e.g., \pt) of visible $\tau$ lepton 
decay products at the generator level. The visible $\tau$ lepton (\visTau), if it decays leptonically, 
is defined as the 4-vector of the light charged lepton. In hadronic decays, \visTau is the difference 
between the 4-vector of the $\tau$ lepton and neutrino in the hadronic decay. 
The visible $\tau$ objects are required to pass the offline kinematic selection criteria ($\eta$ and \pt requirements). 
The \genMET variable is defined as the magnitude of the negative vector sum of the \visTau pairs in the transverse plane. 
The 4-vector of the \visTau objects and \genMET are used to calculate the \mt of the \visTau objects and  also the generator-level \mttwo.  
All efficiencies are derived using the SUSY chargino pair production sample. 
The chargino mass is varied from 120 to 500 \GeV and the neutralino mass from 1 to 500 \GeV. 
Table \ref{tbl:EffTauLep}
\begin{table}[!htb]
\begin{center}
\caption{Efficiencies to select a lepton or \Tau in different channels. Here, $\Tau^1$ and $\Tau^2$ stand for leading and subleading (in \pt) \Tau in the \tauTau channel.} 
\begin{tabular}{|c|c|c|c|c|c|}
\hline
\pt($\ell$  or  $\visTau$) (GeV)         & e for $\eTau$ &  $\mu$ for $\muTau$  & \Tau for $\ell\Tau$    &  $\Tau^1$ for \tauTau & $\Tau^2$ for \tauTau\\
\hline\hline
%0-10                      &    0.15       &    0.01              &         0.001          &       0.0             & 0.52 \\\hline
%10-20                     &    0.14       &    0.72              &         0.004          &       0.0             & 0.54 \\\hline
20--30                     &    0.27       &    0.80              &         0.16           &       $>$ 0.01             & $>$ 0.01 \\\hline
30--40                     &    0.68       &    0.86              &         0.29           &       $>$ 0.01             & $>$ 0.01 \\\hline
40--60                     &    0.75       &    0.87              &         0.34           &       0.03            & 0.61 \\\hline
60--80                     &    0.80       &    0.89              &         0.38           &       0.10            & 0.69 \\\hline
80--120                    &    0.83       &    0.90              &         0.40           &       0.18            & 0.70 \\\hline
120--160                   &    0.86       &    0.90              &         0.41           &       0.22            & 0.70 \\\hline
160--200                   &    0.87       &    0.91              &         0.41           &       0.24            & 0.71 \\\hline
$>$ 200                   &    0.89       &    0.92              &         0.41           &       0.26            & 0.71 \\\hline
\end{tabular}
\label{tbl:EffTauLep}
\end{center}
\end{table}
shows the efficiencies for selecting a lepton or \Tau for different channels versus \pt(\visTau).  
These efficiencies include the scale factors, and efficiencies of object identification, isolation, and trigger. 
Table \ref{tbl:EffMet}
\begin{table}[!htb]
\begin{center}
\caption{Efficiencies of the \MPT requirement in all channels versus \genMET.}
\begin{tabular}{|c|c|}
\hline
\genMET  (GeV)        & All channels\\
\hline\hline
0--10                   &    0.52 \\\hline
10--20                  &    0.58 \\\hline
20--30                  &    0.68 \\\hline
30--40                  &    0.79 \\\hline
40--50                  &    0.87 \\\hline
50--60                  &    0.93 \\\hline
60--70                  &    0.95 \\\hline
70--80                  &    0.97 \\\hline
80--90                  &    0.98 \\\hline
90--100                 &    0.98 \\\hline
100--120                &    0.99 \\\hline
120--140                &    0.99 \\\hline
140--160                &    0.99 \\\hline
$>$160                  &    1.00 \\\hline
\end{tabular}
\label{tbl:EffMet}
\end{center}
\end{table}
shows the efficiencies in all channels to pass the \MPT $>$ 30 \GeV requirement as a function of the \genMET. 
Table \ref{tbl:EffMass}
\begin{table}[!htb]
\begin{center}
\caption{Efficiencies of the invariant mass requirements in different channels versus generated mass.}
\begin{tabular}{|c|c|c|}
\hline
Generated mass (GeV)  & $\ell\Tau$  &  \tauTau \\
\hline\hline
0--5                  &    $>$ 0.01     &   $>$ 0.01   \\\hline
5--10                 &    0.10     &   $>$ 0.01   \\\hline
10--15                &    0.23     &   0.20   \\\hline
15--20                &    0.97     &   0.90   \\\hline
20--25                &    0.99     &   0.94   \\\hline
25--30                &    1.00     &   0.98   \\\hline
30--35                &    0.99     &   1.00   \\\hline
35--40                &    0.98     &   1.00   \\\hline
40--45                &    0.84     &   0.99   \\\hline
45--50                &    0.16     &   0.95   \\\hline
50--55                &    0.04     &   0.68   \\\hline
55--60                &    0.02     &   0.18   \\\hline
60--65                &    0.01     &   0.06   \\\hline
65--70                &    0.04     &   0.03   \\\hline
70--75                &    0.23     &   0.05   \\\hline
75--80                &    0.78     &   0.15   \\\hline
80--85                &    0.91     &   0.40   \\\hline
85--90                &    0.96     &   0.78   \\\hline
90--95                &    0.97     &   0.92   \\\hline
95--100               &    0.98     &   0.95   \\\hline
100--105              &    1.00     &   0.98   \\\hline
105--110              &    1.00     &   0.99   \\\hline
$>$ 110              &    1.00     &   1.00   \\\hline
\end{tabular}
\label{tbl:EffMass}
\end{center}
\end{table}
shows the efficiencies in different channels to pass the requirement of the reconstructed invariant mass 
versus the invariant mass of the 
\visTau pair (generated mass). The requirements
on the invariant mass of the reconstructed pair are ($>$ 15 \GeV) and ($<$ 45 or $>$ 75 \GeV) for the $\ell\Tau$ channels 
and ($<$ 55 or $>$ 85 \GeV) for the \tauTau channel. 
The efficiencies of the (\mttwo $>$ 90 \GeV) requirement in $\ell\Tau$ signal region and \tauTau \binone are listed in Table \ref{tbl:EffMT2}. 
\begin{table}[!htb]
\begin{center}
\caption{Efficiencies of the \mttwo $>$ 90 \GeV requirement in all channels versus generated \mttwo.}
\begin{tabular}{|c|c|c|}
\hline
Generated \mttwo (GeV)    & $\ell\Tau$  &  \tauTau \binone \\
\hline\hline
0--20                     &    $>$ 0.01   &   $>$ 0.01  \\\hline
20--40                    &    0.002    &   0.01  \\\hline
40--50                    &    0.01     &   0.01  \\\hline
50--60                    &    0.02     &   0.03  \\\hline
60--70                    &    0.05     &   0.07  \\\hline
70--80                    &    0.13     &   0.17  \\\hline
80--90                    &    0.35     &   0.44  \\\hline
90--100                   &    0.65     &   0.73  \\\hline
100--110                  &    0.82     &   0.88  \\\hline
110--120                  &    0.90     &   0.94  \\\hline
120--130                  &    0.93     &   0.97  \\\hline
130--140                  &    0.95     &   0.98  \\\hline
140--160                  &    0.96     &   0.98  \\\hline
160--180                  &    0.97     &   0.99  \\\hline
$>$ 180                   &    0.97     &   1.00  \\\hline
\end{tabular}
\label{tbl:EffMT2}
\end{center}
\end{table}
Table \ref{tbl:EffTauMT}
\begin{table}[!htb]
\begin{center}
\caption{Efficiencies of the \tauMT requirement in $\ell\Tau$ channels versus generated \tauMT.}
\begin{tabular}{|c|c|}
\hline
Generated \tauMT (GeV)  & $\ell\Tau$ \\
\hline\hline
100--125                  &   0.01   \\\hline
125--150                  &   0.03   \\\hline
150--170                  &   0.09   \\\hline
170--190                  &   0.26   \\\hline
190--200                  &   0.51   \\\hline
200--210                  &   0.67   \\\hline
210--230                  &   0.82   \\\hline
230--250                  &   0.91   \\\hline
250--275                  &   0.94   \\\hline
275--300                  &   0.97   \\\hline
$>$ 300                   &   1.00   \\\hline
\end{tabular}
\label{tbl:EffTauMT}
\end{center}
\end{table}
shows the efficiencies in the $\ell\Tau$ channels to pass the \tauMT $>$ 200 \GeV requirement versus generated \tauMT.

In the \tauTau \bintwo, the reconstructed \mttwo is constrained to lie between 40 and 90 \GeV. Table \ref{tbl:EffMT2SR2}
\begin{table}[!htb]
\begin{center}
\caption{Efficiencies of the \mttwo requirement in \tauTau \bintwo versus generated \mttwo.}
\begin{tabular}{|c|c|}
\hline
Generated \mttwo (GeV)  &  \tauTau \bintwo \\
\hline\hline
0--20     & 	0.08  \\\hline
20--40    & 	0.43  \\\hline
40--50    & 	0.75  \\\hline
50--60    & 	0.82  \\\hline
60--70    & 	0.81  \\\hline
70--80    & 	0.72  \\\hline
80--90    & 	0.49  \\\hline
90--100   & 	0.24  \\\hline
100--110  & 	0.11  \\\hline
110--120  & 	0.05  \\\hline
120--130  & 	0.03  \\\hline
130--140  & 	0.02  \\\hline
140--160  & 	0.01  \\\hline
160--180  & 	0.01  \\\hline
$>$ 180  & 	$>$ 0.01  \\\hline
\end{tabular}
\label{tbl:EffMT2SR2}
\end{center}
\end{table}
shows the efficiencies in \tauTau \bintwo to pass the 40 $<$ \mttwo $<$ 90 \GeV requirement versus generated \mttwo. 
The last selection in this channel is
the requirement on \SumMT, which is calculated using the 4-vector of the two \visTau and \genMET. Table \ref{tbl:EffSumMT} 
\begin{table}[!htb]
\begin{center}
\caption{Efficiencies of the \SumMT requirement in \tauTau \bintwo versus the generated \SumMT.}
\begin{tabular}{|c|c|c|}
\hline
Generated \SumMT (GeV)  &  \tauTau \bintwo\\
\hline\hline 
$<$ 80        &  $>$ 0.01  \\\hline
80--180       &  0.16  \\\hline
180--200      &  0.19  \\\hline
200--210      &  0.25  \\\hline
210--220      &  0.30  \\\hline
220--230      &  0.36  \\\hline
230--240      &  0.43  \\\hline
240--250      &  0.52  \\\hline
250--260      &  0.55  \\\hline
260--270      &  0.61  \\\hline
270--280      &  0.67  \\\hline
280--290      &  0.68  \\\hline
290--300      &  0.73  \\\hline
300--320      &  0.76  \\\hline
320--340      &  0.77  \\\hline
340--360      &  0.80  \\\hline
360--380      &  0.81  \\\hline
380--400      &  0.81  \\\hline
$>$ 400       &  0.82  \\\hline
\end{tabular}
\label{tbl:EffSumMT}
\end{center}
\end{table}
shows the efficiencies in \tauTau \bintwo to pass the \SumMT $>$ 250 \GeV requirement versus generated \SumMT.

To use these efficiencies, one needs to multiply the values one after another and combine statistically the 
final value with the values reported in Table \ref{tbl:yieldSysSummary}  statistically, to decide if a signal point is excluded. 
At the generator level, a pair of $\ell\Tau$ or \tauTau is selected, when the \visTau objects pass
the corresponding offline kinematic selection criteria.

The efficiencies are used to reproduce the yields in the SMS plane. The results are in agreement with the yields from the full chain of 
simulation and reconstruction within ~30\%.
A user of these efficiencies should be aware that some assumptions can be
broken close to the diagonal (very low mass difference between chargino and neutralino) and these efficiencies cannot be used. 
This compressed region requires a separate analysis, 
because the mass difference of the parent particle and its decay products is comparable 
to the energy threshold used in this analysis to select the objects.
