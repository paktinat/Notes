\section{Additional information for new model testing} 
\label{sect:model}
In the previous sections, a simplified SUSY model is used to optimize the selection criteria and interpret the results. 
Here, the main efficiencies versus generated values are reported, so that these results can be used in 
an approximate manner to examine new models in a MC generator-level study. 
The number of the passed signal events and its uncertainty that 
can be evaluated by a generator-level study 
should be combined statistically with the results in Table \ref{tbl:yieldSysSummary} to find the upper limit 
on the number of signal events
and decide if a model is excluded or still allowed according to  the analysis presented in this paper.

Efficiencies are provided as a function of the kinematic properties (e.g., \pt) of visible $\tau$ lepton 
decay products at the generator level. The visible $\tau$ lepton (\visTau), if it decays leptonically, 
is defined as the 4-vector of the light charged lepton. In hadronic decays, \visTau is the difference 
between the 4-vector of the $\tau$ lepton and neutrino in the hadronic decay. 
The visible $\tau$ objects are required to pass the offline kinematic selection criteria ($\eta$ and \pt requirements). 
The \genMET variable is defined as the magnitude of the negative vector sum of the \visTau pairs in the transverse plane. 
The 4-vector of the \visTau objects and \genMET are used to calculate the \mt of the \visTau objects and  also the generator-level \mttwo.  
All efficiencies are derived using the SUSY chargino pair production sample. 
The chargino mass is varied from 120 to 500\GeV and the neutralino mass from 1 to 500\GeV. 
Table \ref{tbl:EffTauLep}
\begin{table}[!htb]
\begin{center}
\caption{Efficiencies to select a lepton or \Tau in different channels. Here, $\Tau^1$ and $\Tau^2$ stand for leading and subleading (in \pt) \Tau in the \tauTau channel. Zero for the efficiency shows the region where the generated $\tau$ leptons do not pass the kinematical and geometrical selection cuts.} 
\begin{tabular}{cccccc}
\hline
\pt($\ell$  or  $\visTau$) (GeV)         & e for $\eTau$ &  $\mu$ for $\muTau$  & \Tau for $\ell\Tau$    &  $\Tau^1$ for \tauTau & $\Tau^2$ for \tauTau\\
\hline
%0-10                      &    0.15       &    0.01              &         0.001          &       0.0             & 0.52 \\\hline
%10-20                     &    0.14       &    0.72              &         0.004          &       0.0             & 0.54 \\\hline
20--30                     &    0.27       &    0.80              &         0.20           &       0               & 0    \\
30--40                     &    0.68       &    0.86              &         0.36           &       0               & 0    \\
40--60                     &    0.75       &    0.87              &         0.42           &       0.04            & 0.61 \\
60--80                     &    0.80       &    0.89              &         0.47           &       0.14            & 0.69 \\
80--120                    &    0.83       &    0.90              &         0.50           &       0.26            & 0.70 \\
120--160                   &    0.86       &    0.90              &         0.51           &       0.31            & 0.70 \\
160--200                   &    0.87       &    0.91              &         0.51           &       0.34            & 0.71 \\
$>$ 200                    &    0.89       &    0.92              &         0.51           &       0.37            & 0.71 \\\hline
\end{tabular}
\label{tbl:EffTauLep}
\end{center}
\end{table}
shows the efficiencies for selecting a lepton or \Tau for different channels versus \pt(\visTau).  
These efficiencies include the scale factors, and efficiencies of object identification, isolation, and trigger. 
Table \ref{tbl:EffMet}
\begin{table}[!htb]
\begin{center}
\caption{Efficiencies of the \MPT requirement in all channels versus \genMET.}
\begin{tabular}{cc}
\hline
\genMET  (GeV)        & All channels\\
\hline
0--10                   &    0.52 \\
10--20                  &    0.58 \\
20--30                  &    0.68 \\
30--40                  &    0.79 \\
40--50                  &    0.87 \\
50--60                  &    0.93 \\
60--70                  &    0.95 \\
70--80                  &    0.97 \\
80--90                  &    0.98 \\
90--100                 &    0.98 \\
100--120                &    0.99 \\
120--140                &    0.99 \\
140--160                &    0.99 \\
$>$160                 &    1.00  \\\hline
\end{tabular}
\label{tbl:EffMet}
\end{center}
\end{table}
shows the efficiencies in all channels to pass the \MPT $>$ 30\GeV requirement as a function of the \genMET. 
Table \ref{tbl:EffMass}
\begin{table}[!htb]
\begin{center}
\caption{Efficiencies of the invariant mass requirements in different channels versus generated mass.}
\begin{tabular}{ccc}
\hline
Generated mass (GeV)  & $\ell\Tau$  &  \tauTau \\
\hline
5--10                 &    0.10     &   0   \\
10--15                &    0.23     &   0.20   \\
15--20                &    0.97     &   0.90   \\
20--25                &    0.99     &   0.94   \\
25--30                &    1.00     &   0.98   \\
30--35                &    0.99     &   1.00   \\
35--40                &    0.98     &   1.00   \\
40--45                &    0.84     &   0.99   \\
45--50                &    0.16     &   0.95   \\
50--55                &    0.04     &   0.68   \\
55--60                &    0.02     &   0.18   \\
60--65                &    0.01     &   0.06   \\
65--70                &    0.04     &   0.03   \\
70--75                &    0.23     &   0.05   \\
75--80                &    0.78     &   0.15   \\
80--85                &    0.91     &   0.40   \\
85--90                &    0.96     &   0.78   \\
90--95                &    0.97     &   0.92   \\
95--100               &    0.98     &   0.95   \\
100--105              &    1.00     &   0.98   \\
105--110              &    1.00     &   0.99   \\
$>$ 110               &    1.00     &   1.00   \\\hline
\end{tabular}
\label{tbl:EffMass}
\end{center}
\end{table}
shows the efficiencies in different channels to pass the requirement of the reconstructed invariant mass 
versus the invariant mass of the 
\visTau pair (generated mass). The requirements
on the invariant mass of the reconstructed pair are ($>$ 15\GeV) and ($<$ 45 or $>$ 75\GeV) for the $\ell\Tau$ channels 
and ($<$ 55 or $>$ 85\GeV) for the \tauTau channel. 
The efficiencies of the (\mttwo $>$ 90\GeV) requirement in $\ell\Tau$ signal region and \tauTau \binone are listed in Table \ref{tbl:EffMT2}. 
\begin{table}[!htb]
\begin{center}
\caption{Efficiencies of the \mttwo $>$ 90\GeV requirement in all channels versus generated \mttwo.}
\begin{tabular}{ccc}
\hline
Generated \mttwo (GeV)    & $\ell\Tau$  &  \tauTau \binone \\
\hline
20--40                    &    0.002    &   0.01  \\
40--50                    &    0.01     &   0.01  \\
50--60                    &    0.02     &   0.03  \\
60--70                    &    0.05     &   0.07  \\
70--80                    &    0.13     &   0.17  \\
80--90                    &    0.35     &   0.44  \\
90--100                   &    0.65     &   0.73  \\
100--110                  &    0.82     &   0.88  \\
110--120                  &    0.90     &   0.94  \\
120--130                  &    0.93     &   0.97  \\
130--140                  &    0.95     &   0.98  \\
140--160                  &    0.96     &   0.98  \\
160--180                  &    0.97     &   0.99  \\
$>$ 180                  &    0.97     &   1.00  \\\hline
\end{tabular}
\label{tbl:EffMT2}
\end{center}
\end{table}
Table \ref{tbl:EffTauMT}
\begin{table}[!htb]
\begin{center}
\caption{Efficiencies of the \tauMT requirement in $\ell\Tau$ channels versus generated \tauMT.}
\begin{tabular}{cc}
\hline
Generated \tauMT (GeV)  & $\ell\Tau$ \\
\hline
100--125                  &   0.01   \\
125--150                  &   0.03   \\
150--170                  &   0.09   \\
170--190                  &   0.26   \\
190--200                  &   0.51   \\
200--210                  &   0.67   \\
210--230                  &   0.82   \\
230--250                  &   0.91   \\
250--275                  &   0.94   \\
275--300                  &   0.97   \\
$>$ 300                  &   1.00   \\\hline
\end{tabular}
\label{tbl:EffTauMT}
\end{center}
\end{table}
shows the efficiencies in the $\ell\Tau$ channels to pass the \tauMT $>$ 200\GeV requirement versus generated \tauMT.

In the \tauTau \bintwo, the reconstructed \mttwo is constrained to lie between 40 and 90\GeV. Table \ref{tbl:EffMT2SR2}
\begin{table}[!htb]
\begin{center}
\caption{Efficiencies of the \mttwo requirement in \tauTau \bintwo versus generated \mttwo. Zero for the efficiency shows the region that the generated 
\mttwo is much greater than the selection cut.}
\begin{tabular}{cc}
\hline
Generated \mttwo (GeV)  &  \tauTau \bintwo \\
\hline
0--20     & 	0.08  \\
20--40    & 	0.43  \\
40--50    & 	0.75  \\
50--60    & 	0.82  \\
60--70    & 	0.81  \\
70--80    & 	0.72  \\
80--90    & 	0.49  \\
90--100   & 	0.24  \\
100--110  & 	0.11  \\
110--120  & 	0.05  \\
120--130  & 	0.03  \\
130--140  & 	0.02  \\
140--160  & 	0.01  \\
160--180  & 	0.01  \\
$>$ 180  & 	0  \\\hline
\end{tabular}
\label{tbl:EffMT2SR2}
\end{center}
\end{table}
shows the efficiencies in \tauTau \bintwo to pass the 40 $<$ \mttwo $<$ 90\GeV requirement versus generated \mttwo. 
The last selection in this channel is
the requirement on \SumMT, which is calculated using the 4-vector of the two \visTau and \genMET. Table \ref{tbl:EffSumMT} 
\begin{table}[!htb]
\begin{center}
\caption{Efficiencies of the \SumMT requirement in \tauTau \bintwo versus the generated \SumMT.}
\begin{tabular}{ccc}
\hline
Generated \SumMT (GeV)  &  \tauTau \bintwo\\
\hline 
80--180       &  0.16  \\
180--200      &  0.19  \\
200--210      &  0.25  \\
210--220      &  0.30  \\
220--230      &  0.36  \\
230--240      &  0.43  \\
240--250      &  0.52  \\
250--260      &  0.55  \\
260--270      &  0.61  \\
270--280      &  0.67  \\
280--290      &  0.68  \\
290--300      &  0.73  \\
300--320      &  0.76  \\
320--340      &  0.77  \\
340--360      &  0.80  \\
360--380      &  0.81  \\
380--400      &  0.81  \\
$>$ 400      &  0.82  \\\hline
\end{tabular}
\label{tbl:EffSumMT}
\end{center}
\end{table}
shows the efficiencies in \tauTau \bintwo to pass the \SumMT $>$ 250\GeV requirement versus generated \SumMT.

To take into account the inefficiencies and misidentifications for charge reconstruction of the
objects, identification of the b-tagged jets, identification of the extra leptons and the minimum angle between
the jets and \MET in the transverse plane, the final yields in \leptonTau and \tauTau 
channels must be multiplied by 0.8 and 0.7, respectively.

To use these efficiencies, one needs to multiply the values one after another and combine statistically the 
final value with the values reported in Table \ref{tbl:yieldSysSummary}  statistically, to decide if a signal point is excluded. 
At the generator level, a pair of $\ell\Tau$ or \tauTau is selected, when the \visTau objects pass
the corresponding offline kinematic selection criteria.

The efficiencies are used to reproduce the yields in the SMS plane. The results are in agreement with the yields from the full chain of 
simulation and reconstruction within ~30\%.
A user of these efficiencies should be aware that some assumptions can be
broken close to the diagonal (very low mass difference between chargino and neutralino) and these efficiencies cannot be used. 
This compressed region requires a separate analysis, 
because the mass difference of the parent particle and its decay products is comparable 
to the energy threshold used in this analysis to select the objects.
