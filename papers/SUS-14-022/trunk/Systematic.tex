\section{Systematic uncertainties}
\label{sect:sys}
Systematic uncertainties can affect the shape or normalization of the
backgrounds estimated from simulation (\ttbar, Z+jets, diboson, and Higgs boson events), 
as well as the signal acceptance. 
Systematic uncertainties of other background contributions are described in Sections \ref{sect:bkgQCD}, \ref{sect:bkgW} and \ref{sect:bkgFake}.
The uncertainties are listed below, and summarized in Table~\ref{Tab.SYS}.

\begin{itemize}


\item  The energy scales for electron, muon, and \Tau objects affect the shape of the kinematic distributions.
 The systematic uncertainties in the muon and electron energy scales are negligible.
The visible energy of \Tau object in the MC simulation is scaled up and down
by 3\%, and all \Tau-related variables are recalculated. The resulting variations in
final yields are taken as the systematic uncertainties. They are evaluated to be 10--15\% for 
backgrounds and 2--15\% in different parts of the signal phase space.

\item The uncertainty in the \Tau identification efficiency is 6\%. The uncertainty in the trigger 
efficiency of the \Tau part of the \eTau and \muTau (\tauTau) triggers amounts to 3.0\% (4.5\%) per 
\Tau candidate. A ``tag-and-probe'' technique \cite{Chatrchyan:2014mua} on $\cPZ\to \Pgt\Pgt$ data 
events is used to estimate these uncertainties \cite{Khachatryan:2014wca}.

\item The uncertainty in electron and muon trigger, identification, and
isolation efficiencies is 2\% \cite{Khachatryan:2014wca}.

\item The uncertainty due to the scale factor for the b-tagging efficiency and misidentification 
rate is evaluated by varying the factors within their uncertainties. The yields of signal and background events are changed by 8\% 
and 4\%, respectively \cite{Chatrchyan:2012jua}.
 
\item To evaluate the uncertainty due to pileup, the measured inelastic pp cross section is
  varied by 5\% \cite{Antchev:2011vs}, resulting in a change in the number of simulated pileup interactions.
 The relevant efficiencies for signal and background events are changed by 4\%.

\item The uncertainty in the signal acceptance due to parton distribution function (PDF) uncertainties 
  is taken to be 2\% from a similar analysis \cite{Khachatryan:2014qwa} which follows the PDF4LHC recommendations \cite{pdf4lhc}.

\item The uncertainty in the integrated luminosity  is $2.6\%$ \cite{CMS-PAS-LUM-13-001}.  This affects only the
  normalization of the signal MC samples. Because for the backgrounds  either control samples in data are used or the normalization is measured from data.

\item The uncertainty in the signal acceptance associated with initial-state radiation (ISR)
is evaluated by comparing the efficiencies of jet-related requirements  
in the \MADGRAPH+ \PYTHIA program. 
Using the SM WW process, which
 is expected to be similar to chargino pair production in terms of parton content and process, a 3\% uncertainty in 
the efficiency of  b-tagged jets veto and a 6\% uncertainty in the \deltaphi requirement are assigned.

\item The uncertainties related to \MPT can arise from different sources, e.g.  the energy scales of lepton, \Tau, and jet 
objects, and unclustered energy.  The unclustered energy is the energy of the reconstructed objects which
 do not belong to any jet or lepton with \PT $>$ 10 \GeV. The effect of lepton and \Tau
 energy scales is discussed above. The contribution from the uncertainty in the jet energy scale (2--10\% depending on $\eta$  and \PT) and
 unclustered energy (10\%) is found to be negligible. A conservative value of 5\% uncertainty
 is assigned to both signal and background processes based on MC simulation studies \cite{Khachatryan:2015kxa, Khachatryan:2014qwa}.

\item The performance of the fast detector simulation has some differences compared to the full detector simulation, especially in
 track reconstruction \cite{Khachatryan:2015kxa} that can affect the \Tau isolation. A 5\% systematic uncertainty per
 \Tau candidate is assigned by comparing the \Tau isolation and identification efficiency in the fast
 and full simulations. 

\item The statistical uncertainties due to limited numbers of simulated events also contributes to the overall uncertainties. 
This uncertainty amounts to 3--15\% for the different parts of the signal phase space and 13--70\% for the backgrounds in different signal regions.

\item For less important backgrounds like \ttbar,  dibosons, and Higgs boson production, the number of simulated 
events remaining after event selection is very small. A 50\% uncertainty is considered for these backgrounds 
to account for the possible theoretical uncertainty in the cross section calculation as well as the shape mismodeling.
\end{itemize}

\begin{table}[!htb]
\begin{center}
\caption{Summary of the systematic uncertainties that affect the signal event 
selection efficiency and the background normalization and their shapes. 
The sources that affect the shape are indicated by (*) next to their names. 
These sources are considered correlated between two signal regions of the 
\tauTau analysis in the final statistical combination.}
\small{
\begin{tabular}{|l|ccc|ccc|}
\hline
                              &\multicolumn{3}{c|}{Background (\%)}         &\multicolumn{3}{c|}{Signal (\%)}\\\hline
                              &            & \tauTau & \tauTau         &            & \tauTau & \tauTau\\
Systematic uncertainty source & \leptonTau & \binone &  \bintwo        & \leptonTau & \binone &  \bintwo        \\
\hline\hline
\Tau energy scale (*)          &10 &\multicolumn{2}{c|}{15}  & 2--12 &\multicolumn{2}{c|}{3--15} \\%\hline 
\Tau identification efficiency & 6 &\multicolumn{2}{c|}{12} & 6 &\multicolumn{2}{c|}{12}  \\%\hline
\Tau trigger  efficiency       & 3&\multicolumn{2}{c|}{9}& 3&\multicolumn{2}{c|}{9}  \\%\hline
Lepton trigger and ident. eff. & 2 & \multicolumn{2}{c|}{--} & 2 &  \multicolumn{2}{c|}{--} \\%\hline
b-tagged jets veto              & 4 & -- & 4 &  8 & -- & 8 \\%\hline
Pileup&\multicolumn{3}{c|}{4} &\multicolumn{3}{c|}{4} \\%\hline
PDF (*)&\multicolumn{3}{c|}{--}&\multicolumn{3}{c|}{2} \\%\hline
Integrated luminosity       &\multicolumn{3}{c|}{--} & \multicolumn{3}{c|}{2.6}\\%\hline
ISR (*)&\multicolumn{3}{c|}{--}&\multicolumn{3}{c|}{3} \\%\hline
\mindphifour&\multicolumn{3}{c|}{--}&\multicolumn{3}{c|}{6} \\%\hline
\MPT (*)&\multicolumn{3}{c|}{5} &\multicolumn{3}{c|}{5} \\%\hline
Fast/full \Tau ident. eff. &\multicolumn{3}{c|}{--}& 5 & \multicolumn{2}{c|}{10}\\\hline
Total shape-affecting sys. & 11 & 16 & 16 & 6--13 &\multicolumn{2}{c|}{7--16} \\\hline
Total non-shape-affecting sys. & 9 & 16 & 16 & 14 &20& 21 \\\hline
Total systematic &  14 & 22  & 22& 15--19 & 21--25  & 22--26\\\hline
MC statistics & 22 & 13 & 70 & \multicolumn{3}{c|}{3--15} \\\hline
Total& 26 & 26  & 73& 15--24 & 21--29  & 22--30\\\hline
Low-rate backgrounds &\multicolumn{3}{c|}{50}&\multicolumn{3}{c|}{--}\\\hline

\end{tabular}
}
\label{Tab.SYS}
\end{center}
\end{table}


\noindent The systematic uncertainties that can alter the shapes are added in quadrature and 
treated as correlated when two signal regions of the \tauTau channel are combined. Other systematic uncertainties of these two 
channels and all of the systematic uncertainties of the \leptonTau channels are treated as uncorrelated.


