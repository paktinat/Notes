\section{Results and interpretation}
\label{sect:stat}
The observed data and predicted background yields for the four signal regions are summarized in Table~\ref{tbl:yieldSysSummary}. 
\begin{table}[!htb]
\begin{center}
\begin{small}
\caption{Data yields and background predictions with uncertainties in the four signal regions of the search. 
The uncertainties are reported in two parts, the statistical and systematic uncertainties, respectively. 
The \wjets and QCD multijet main backgrounds are derived from data as described in Section~\ref{sect:bkg}; 
the abbreviation ``VV'' refers to diboson events. The yields for three signal points representing the low, medium, and high $\Delta m$
are also shown. SUSY(X, Y) stands for a SUSY signal with ${\rm{m}}_{\chione}$ = X\GeV and ${\rm{m}}_{\neutralino}$ = Y\GeV.}
\begin{tabular}{|c|c|c|c|c|}
\hline
	           & \eTau & \muTau & \tauTau \binone & \tauTau \bintwo \\
\hline
  DY               & 0.19 $\pm$ 0.04 $\pm$ 0.03 & 0.25 $\pm$ 0.06  $\pm$ 0.04  &  0.56 $\pm$ 0.07 $\pm$ 0.12 & 0.81 $\pm$ 0.56 $\pm$ 0.18  \\
tX, VV, hX  & 0.03 $\pm$ 0.03 $\pm$ 0.02 & 0.19 $\pm$ 0.09  $\pm$ 0.09  &  0.19 $\pm$ 0.03 $\pm$ 0.09 & 0.75 $\pm$ 0.35 $\pm$ 0.38  \\
\wjets             & 3.30$_{- 3.30}^{+ 3.35}$ $\pm$ 0.56 & 8.15 $\pm$ 4.59  $\pm$ 1.53  &  0.70 $\pm$ 0.21 $\pm$ 0.55 & 4.36 $\pm$ 1.05 $\pm$ 1.63  \\
QCD multijet       &             -              &            -                 &  0.13 $\pm$ 0.06 $\pm$ 0.21 & 1.15 $\pm$ 0.39 $\pm$ 0.74  \\
\hline
SM total           & 3.52 $\pm$ 3.35 $\pm$ 0.56 & 8.59 $\pm$ 4.59  $\pm$ 1.53  &  1.58 $\pm$ 0.23 $\pm$ 0.61 & 7.07 $\pm$ 1.30 $\pm$ 1.84  \\
\hline
Observed           &               3            &                5             &             1               & 2     \\\hline  
SUSY(380, 1)        & 2.14 $\pm$ 0.08 $\pm$ 0.38 & 2.16 $\pm$ 0.08  $\pm$ 0.39  &  4.10 $\pm$ 0.10 $\pm$ 0.90 & 1.10 $\pm$ 0.05 $\pm$ 0.27 \\
SUSY(240, 40)       & 1.43 $\pm$ 0.19 $\pm$ 0.21 & 0.96 $\pm$ 0.14  $\pm$ 0.14  &  4.35 $\pm$ 0.27 $\pm$ 0.91 & 3.60 $\pm$ 0.25 $\pm$ 0.83 \\
SUSY(180, 60)       & 0.12 $\pm$ 0.04 $\pm$ 0.02 & 0.04 $\pm$ 0.02  $\pm$ 0.01  &  0.73 $\pm$ 0.11 $\pm$ 0.17 & 2.36 $\pm$ 0.17 $\pm$ 0.54 \\
\hline
\end{tabular}
\label{tbl:yieldSysSummary}
\end{small}
\end{center}
\end{table}

In all signal regions the observed data  are consistent with the predicted SM values within the uncertainties. 
%Even in the signal region of \tauTau \bintwo, the observed data is closer than 2 times of the total uncertainty to the central 
%value of the SM prediction.

Figure \ref{fig:yield_final}
\begin{figure}[!htb]
\centering
\includegraphics[width=0.475\textwidth,keepaspectratio=true]{StatisticsFig/MT2_tauMTgt200_DDFakeEleTau.pdf}
\includegraphics[width=0.475\textwidth,keepaspectratio=true]{StatisticsFig/MT2muTau_tauMTgt200_DDFake.pdf}
\includegraphics[width=0.475\textwidth,keepaspectratio=true]{StatisticsFig/QCDWestimation_bin1.pdf}
\includegraphics[width=0.475\textwidth,keepaspectratio=true]{StatisticsFig/QCDWestimation_bin2.pdf}
\caption{The data yield is compared with the SM expectation. In different signal regions, 
when a data driven background is available, it is used instead of the pure simulation. For more details, see the text.}
\label{fig:yield_final}
\end{figure}
compares the data and the SM expectation in four search regions. The top row 
shows the \mttwo distributions in the \leptonTau channels. 
In these plots, the QCD multijet and \wjets and fake contribution from other channels 
are based on the estimate described in Section \ref{sect:bkgFake} and labeled ``W''.
The QCD multijet contribution is very low for these channels and is counted among ``W''.
The bottom row shows the \mttwo and \SumMT distributions in two different signal regions of \tauTau channel. 
The QCD multijet contribution in these plots is obtained using the data driven method described in 
Section \ref{sect:bkgQCD}. The \wjets contribution in 
the last bin of the bottom plots is described in Section \ref{sect:bkgW}, while the other bins are based on simulated events.
The uncertainty band in these four plots includes both the statistical and systematic uncertainties.

There is no excess of events over the SM expectation.  We interpret our results in the context
of a simplified model of chargino pair production and decay, which is described in Section~\ref{sect:MCSamples} and corresponds 
to the left diagram in Fig.~\ref{fig:Productions}. 

A modified frequentist approach, known as the CLs method \cite{read:CLs}, is used to 
set limits on cross sections at 95\% confidence level.
Combining all four signal regions,
the observed limits rule out \chione  masses up to  421\GeV  for a massless \PSGczDo.  
%and  \PSGczDo masses up to 110\GeV for some  \chione masses, see Fig.~\ref{fig:limit_final}. 
The results on excluded regions are shown in Fig.~\ref{fig:limit_final}. 
\begin{linenomath}
\begin{figure}[!htb]
\centering
\includegraphics[width=0.7\textwidth,keepaspectratio=true]{StatisticsFig/Exclusion4Bins.pdf}
\caption{Expected and observed exclusion regions in terms of Simplified Models of
chargino pair production 
with the total dataset of 2012. The bottom-left triangle was excluded by LEP \sTau searches. 
The diagonal line denotes the boundary for $m_{\chione} = m_{\tau} + m_{\neutralino}$.
The $\pm$ 1 standard deviations of the expected (observed) exclusions introduced by the experimental 
(theoretical) uncertainties are also shown.}
\label{fig:limit_final}
\end{figure}
\end{linenomath}
This should be compared to the ATLAS limit of 345\GeV \cite{Aad:2014yka}.
It should be noted that the ATLAS results are based on the \tauTau channel. Figure 
\ref{fig:limit_tauTau} 
\begin{linenomath}
\begin{figure}[!htb]
\centering
\includegraphics[width=0.7\textwidth,keepaspectratio=true]{StatisticsFig/ExclusionTauTau2Bin.pdf}
\caption{Expected and observed exclusion regions in terms of Simplified Models
in the \tauTau channel. The conventions are same as Fig.~\ref{fig:limit_final}.}
\label{fig:limit_tauTau}
\end{figure}
\end{linenomath}
shows our results in the \tauTau channel, where the \chione masses are excluded up to 400\GeV for a massless \PSGczDo. 
A better limit is obtained with our selection requirements because the number of SM background events is reduced relative to the previous analysis.

The \sTau searches in the LEP experiments \cite{lepsusy} have excluded masses below 95\GeV. In Fig.~\ref{fig:limit_final} and 
\ref{fig:limit_tauTau}, this region corresponds to the triangle in bottom-left corner. 
The diagonal line denotes the boundary for $m_{\chione} = m_{\tau} + m_{\neutralino}$, which is the kinematical boundary of the search.
The expected limits and the contours corresponding to $\pm$ 1 standard deviations from experimental uncertainties are shown with the red solid and dashed lines, respectively. 
The observed limits are shown with a black solid line, 
while the $\pm$ 1 standard deviations based on signal cross section uncertainties are shown with narrower black lines.
The signal cross sections in NLO + NLL order in $\alpha_s$ are used to make the exclusion limits.
In the whole region, the observed limits are within one standard deviation from the expected limits.  


The results of the \tauTau channels are also interpreted to set limits on \sTau\sTau production, 
which corresponds to the right diagram in Fig.~\ref{fig:Productions}. 
In this simplified model, two \sTau particles are directly produced from the $pp$ collision and decay instantly 
into two $\tau$ leptons and two neutralinos. 
The two $\ell\Tau$ channels are not considered in this interpretation, because they do not improve the results. 
To calculate the production cross section, \sTau is 
defined as a maximal admixture of the left-handed and right-handed \sTau gauge-eigenstates \cite{Fuks:2013lya}. 
Since the cross section for direct production of sleptons is lower, no point is excluded and a $95\%$ upper limit is set on 
the cross section  as a function of the \sTau mass. 
Figure \ref{fig:limit_stau_stau} represents the ratio of the 
\begin{linenomath}
\begin{figure}[!htb]
\centering
\includegraphics[width=0.7\textwidth,keepaspectratio=true]{StatisticsFig/ExclusionSTauSTauLsp1.pdf}
\caption{Upper limits on \sTau\sTau production cross section in the \tauTau channel. The mass of \PSGczDo is 1\GeV.}
\label{fig:limit_stau_stau}
\end{figure}
\end{linenomath}
obtained upper limit on the cross section and the cross section expected from SUSY (signal strength) vs. the mass of the \sTau particle, with the \PSGczDo mass set to 1\GeV.
The observed ratio is within one standard deviation of  the expected ratio.
The best limit, which corresponds to the lowest signal strength, is obtained for $m_{\sTau}=150\GeV$. The observed (expected) upper limit on the cross section at this mass is 43 (56) fb which is almost two  times larger than the theoretical NLO cross section.



