\section{\texorpdfstring{Event selection for the \tauTau channel}{Event selection for the tau-tau channel}}
\label{sect:tauTauCuts}
In this channel data of proton-proton collisions,  corresponding to an integrated luminosity of 18.1 $\mathrm{fb}^{-1}$, are used.
The events are first selected with a trigger \cite{Chatrchyan:2011nv} that requires the presence of two isolated 
\Tau candidates with \PT $>$ 35\GeV and $|\eta|<$ 2.1, passing loose identification requirements.

Offline, the two \Tau candidates must pass the tight $\tau$ isolation discriminator,
\PT $>$ 45\GeV and $|\eta|<$ 2.1, and have opposite sign (OS).
In events with more than one \tauTau pair, only the pair with the most isolated \Tau objects is considered. 

Events with extra isolated electrons or muons of \PT $>$ 10\GeV and $|\eta| <$ 2.4 
are rejected to suppress backgrounds from diboson decays. 
Inspired from the MC studies, 
the contribution from the \Z$ \rightarrow$ \tauTau  background is reduced by rejecting events 
where the visible di-\Tau invariant mass is between 55 and 85\GeV (\Z boson veto).  
Furthermore, contributions from low-mass DY and QCD multijet production are 
reduced by requiring the invariant mass to be greater than 15\GeV.
To further reduce \Z $\rightarrow$ \tauTau and QCD multijet events, 
\MPT $>$ 30\GeV and \mttwo $>$ 40\GeV are also required.
The minimum angle \deltaphi in the transverse plane between the \ptvecmiss and any of the \Tau and jets, 
including b-tagged jets, must be greater than 1.0 radians. 
This requirement reduces backgrounds from QCD multijet events and \wjets events.

After applying the preselection described above,
additional requirements are introduced to define two search regions.
The first search region (\binone) targets models with a large mass difference ($\Delta m$) 
between charginos and neutralinos.
In this case, the \mttwo signal distribution can have a long tail beyond the 
distribution of SM backgrounds.
The second search region (\bintwo) is dedicated to models with small values of $\Delta m$.
In this case, the sum of the two transverse mass values, \SumMT = $\mt(\Tau^1,\ptvecmiss) + \mt(\Tau^2,\ptvecmiss)$, 
provides additional discrimination between signal and SM background processes.

The two signal regions (SR) are defined as:
\begin{itemize}
\item {\bf \binone}: \mttwo $>$ 90\GeV;
\item {\bf \bintwo}:  \mttwo $<$ 90\GeV, \SumMT $>$ 250\GeV, and b-tagged jets are vetoed.
\end{itemize}
The veto on b-tagged jets in SR2 reduces the
number of \ttbar events, which
are expected in  the low-\mttwo region. Table \ref{Tab.Cuts} summarizes the selection requirements for the different signal regions.
\begin{table}[!htb]
\begin{center}
\caption{Definition of the signal regions.}
\begin{tabular}{|c|c|c|}
\hline
               & \tauTau & \tauTau               \\
   \leptonTau  & \binone & \bintwo               \\\hline\hline
 OS \leptonTau & \multicolumn{2}{c|}{OS \tauTau}  \\\hline
\multicolumn{3}{|c|}{Extra lepton veto}          \\\hline
\multicolumn{3}{|c|}{Invariant mass of \leptonTau or \tauTau $>$ 15\GeV}\\\hline
\multicolumn{3}{|c|}{\Z boson mass veto}              \\\hline
\multicolumn{3}{|c|}{\MPT $>$ 30\GeV}            \\\hline
\multicolumn{3}{|c|}{\deltaphi $> 1.0 $ radians}         \\\hline
\multicolumn{3}{|c|}{$\mttwo > 40\GeV$}         \\\hline\hline
b-tagged jet veto&  - & b-tagged jet veto  \\\hline
\multicolumn{2}{|c|}{$\mttwo > 90\GeV$} & $\mttwo < 90\GeV$ \\\hline
$\tauMT > 200\GeV$    &  - & $\SumMT > 250\GeV$ \\\hline
\end{tabular}
\label{Tab.Cuts}
\end{center}
\end{table}
